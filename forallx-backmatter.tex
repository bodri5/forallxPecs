%!TEX root = forallx.tex
\thispagestyle{empty}
\onecolumn
\ 
\vfill

\parbox{3 in}{
In the Introduction to his volume \emph{Symbolic Logic}, Charles Lutwidge Dodson advised: ``When you come to any passage you don't understand, \emph{read it again}: if you \emph{still} don't understand it, \emph{read it again}: if you fail, even after \emph{three} readings, very likely your brain is getting a little tired. In that case, put the book away, and take to other occupations, and next day, when you come to it fresh, you will very likely find that it is \emph{quite} easy.''

A Szimbolikus Logika kötet bevezetésében Charles Lutwidge Dodson azt tanácsolta: „Ha olyan résznél jársz, amit nem értesz, olvasd el újra: ha továbbra sem érted, olvasd el újra: ha még három olvasás után is elbuksz, akkor valószínűleg az agyad kicsit elfáradt. Ebben az esetben tedd el a könyvet, és foglalkozz valami mással, majd másnap felfrissülten nagy valószínűséggel egyszerűnek találod majd.”
\medskip

The same might be said for this volume, although readers are forgiven if they take a break for snacks after \emph{two} readings.

Hasonló mondható el erről a kötetről is, habár megbocsátható, ha az olvasó nasi szünetet tart két olvasás között.
}

\vfill

\parbox{3 in}{
{\sf about the author:}
\medskip

P.D. Magnus is a professor of philosophy in Albany, New York. His primary research is in the philosophy of science.

{\sf A szerzőről:}
\medskip

P.D. Magnus filozófia professzor a New York állambeli Albanyben. Főbb kutatási területe a tudomány filozófiája.
}
\vfill

%PK fordítása vége