%!TEX root = forallx.tex

%CE fordítása kezdet

\chapter*{Proofs}
\chapter{Bizonyítások}
\label{ch.proofs}

Consider two arguments in SL:

Vegyünk két érvelést KL-ban:

\begin{multicols}{2}
Argument A\\
A érvelés
\begin{earg}
\item[] $P \eor Q$
\item[] $\enot P$
\item[\therefore] Q
\end{earg}

Argument B \\
B érvelés
\begin{earg}
\item[] $P \eif Q$
\item[] $P$
\item[\therefore] Q
\end{earg}

\end{multicols}

Clearly, these are valid arguments. You can confirm that they are valid by constructing four-line truth tables. Argument A makes use of an inference form that is always valid: Given a disjunction and the negation of one of the disjuncts, the other disjunct follows as a valid consequence. This rule is called \emph{disjunctive syllogism}.

Tisztán láthatóan ezek helyes érvelések. Be lehet bizonyítani a helyességüket négy soros igazságtáblák segítségével. Az A érvelés egy olyan következtetési formát használ, ami mindig igaz: Ha van egy diszjunkciónk és egy diszjunkt tagadása, akkor a másik diszjunkt egy igaz következmény. Ezt a szabályt (tételt) nevezzük \emph{diszjunktív szillogizmusnak}.

Argument B makes use of a different valid form: Given a conditional and its antecedent, the consequent follows as a valid consequence. This is called \emph{modus ponens}.

A B érvelés egy másik érvelési formát használ: Legyen egy implikációnk, és annak premisszája. Akkor a következmény egy igaz következmény. Ezt a szabályt hívjuk \emph{modus ponensnek}.

 When we construct truth tables, we do not need to give names to different inference forms. There is no reason to distinguish modus ponens from a disjunctive syllogism. For this same reason, however, the method of truth tables does not clearly show \emph{why} an argument is valid. If you were to do a 1024-line truth table for an argument that contains ten sentence letters, then you could check to see if there were any lines on which the premises were all true and the conclusion were false. If you did not see such a line and provided you made no mistakes in constructing the table, then you would know that the argument was valid. Yet you would not be able to say anything further about why this particular argument was a valid argument form.
 
Amikor létrehozzuk az igazságtáblákat, akkor nem kell különböző neveket adnunk a következtetési formáknak. Nincs oka, hogy megkülönböztessük a modus ponens-t a diszjunktív szillogizmustól. De pont ugyanezért az igazságtáblák nem mutatják pontosan hogy  \emph{miért} igaz egy érvelés. Ha csinálnál egy 1024 soros igazságtáblát egy állításához, ami tíz mondatbetűt tartalmaz, akkor meg tudnád nézni, hogy voltak-e olyan sorok ahol a premisszák mind igazak voltak és a következtetés hamis. Ha nem láttál olyan sort, feltételezve, hogy mindent helyesen csináltál az igazságtábla létrehozásakor, akkor tudnád hogy az érvelés helyes. De mégse tudnál többet mondani, hogy ez miért egy helyes érvelési forma. 

The aim of a \emph{proof system} is to show that particular arguments are valid in a way that allows us to understand the reasoning involved in the argument. We begin with basic argument forms, like disjunctive syllogism and modus ponens. These forms can then be combined to make more complicated arguments, like this one:

Egy \emph{bebizonyítási rendszernek} az a célja, hogy megmutassa az egyes érvelések helyességét, egy olyan módon, hogy megértsük az okot az érvelésben. Egyszerű érvelési formákkal kezdünk, mint a diszjunktív szillogizmus és a modus ponens. Ezek a formák egyesíthetőek lesznek majd, hogy még bonyurultabb érveléseket hozzunk létre, mint például ezt:

%CE fordítása vége

%AE fordítása kezdet

\begin{earg}
\item[(1)] $\enot L \eif (J \eor L)$
\item[(2)] $\enot L$
\item[\therefore] $J$
\end{earg}
By modus ponens, (1) and (2) entail $J \eor L$. This is an \emph{intermediate conclusion}. It follows logically from the premises, but it is not the conclusion we want. Now $J \eor L$ and (2) entail $J$, by disjunctive syllogism. We do not need a new rule for this argument. The proof of the argument shows that it is really just a combination of rules we have already introduced.

A modus ponens miatt, (1) és (2)-ből következik, hogy $J \eor L$. Ami egy \emph{közbülső következtetés}. A premisszákból logikusan következik, de ez még nem az a konklúzió, amit szeretnénk megkapni. $J \eor L$-ből és (2)-ből következik $J$ a diszjunktív szillogizmus miatt. Az érveléshez nincs szükségünk új szabályra. Az érvelés bizonyítása megmutatja, hogy tényleg azoknak a szabályoknak egy kombinációja, amik fentebb lettek bevezetve.

Formally, a \define{proof} is a sequence of sentences. The first sentences of the sequence are assumptions; these are the premises of the argument. Every sentence later in the sequence follows from earlier sentences by one of the rules of proof. The final sentence of the sequence is the conclusion of the argument.

Formálisan egy \define{bizonyítás} mondatok sora. A következtetés első mondatai a hipotézisek; amelyek az érvelés premisszái. A következtetés minden későbbi mondata egy vagy több korábbi mondatból következik a bizonyítás egyik szabálya által. A következtetés utolsó mondata az érvelés konklúziója.

This chapter begins with a proof system for SL, which is then extended to cover QL and QL plus identity.

Ez a fejezet a KL bizonyításrendszerével kezdődik, ami a későbbiekben ki lesz terjesztve a PL-re és a PL plusz identitására.

\section*{Basic rules for SL}
\section{Alapszabályok KL-re}

In designing a proof system, we could just start with disjunctive syllogism and modus ponens. Whenever we discovered a valid argument which could not be proven with rules we already had, we could introduce new rules. Proceeding in this way, we would have an unsystematic grab bag of rules. We might accidently add some strange rules, and we would surely end up with more rules than we need.

Egy bizonyításrendszer létrehozásakor nyugodtan kezdhetünk a diszjunktív szillogizmussal és a modus ponensszel. Amikor találunk egy olyan érvet, amit nem lehet megadni a rendelkezésünkre álló szabályok segítségével, új szabályokat vezethetünk be. Ilyen módon a szabályoknak egy nem szisztematikusan kapott halmaza állna rendelkezésünkre. Lehet, hogy véletlenül hozzáadtunk néhány furcsa szabályt, és a végére így biztosan a szükségesnél több szabályunk lenne.

Instead, we will develop what is called a \define{natural deduction} system. In a natural deduction system, there will be two rules for each logical operator: an \define{introduction} rule that allows us to prove a sentence that has it as the main logical operator and an \define{elimination} rule that allows us to prove something given a sentence that has it as the main logical operator.

Ehelyett, egy \define{természetes levezetési, avagy dedukciós} rendszert használunk. Egy természetes levezetési rendszerben 2 szabály van minden logikai operátorra: egy \define{bevezető} szabály, amely lehetővé teszi számunkra, hogy bebizonyítsunk egy mondatot, amelyben ez a fő logikai operátor és egy \define{elimináló} szabály, amely lehetővé teszi számunkra, hogy bebizonyítsunk valamit egy mondattal, aminek ez a fő logikai operátora.

In addition to the rules for each logical operator, we will also have a reiteration rule. If you already have shown something in the course of a proof, the reiteration rule allows you to repeat it on a new line. For instance:

A logikai operátorokhoz tartozó szabályokon túl lesz egy ismétlési szabályunk is. Ha már a bizonyítás során bebizonyítottunk valamit, akkor az ismétlési szabály lehetővé teszi számunkra, hogy megismételjük egy új sorban. Például:

\begin{proof}
	\have{a1}{\script{A}}
	\have{a2}{\script{A}} \by{R}{a1}
\end{proof}

When we add a line to a proof, we write the rule that justifies that line. We also write the numbers of the lines to which the rule was applied. The reiteration rule above is justified by one line, the line that you are reiterating. So the `R 1' on line 2 of the proof means that the line is justified by the reiteration rule (R) applied to line 1.

Amikor hozzáadunk egy új sort a bizonyításhoz, leírjuk azt a szabályt, ami igazolja azt a sort. Azoknak a soroknak számát is leírjuk, melyekre a szabály alkalmazva lett. A fenti ismétlési szabályt csak egy sor, a megismételt sor igazolja. Tehát a bizonyítás 2. sorában az `R 1' azt jelenti, hogy a sort az 1. sorra alkalmazott ismételési szabály (R) igazolja.

%AE fordítása vége

%GyRJ fordítása kezdet

Obviously, the reiteration rule will not allow us to show anything \emph{new}. For that, we will need more rules. The remainder of this section will give introduction and elimination rules for all of the sentential connectives. This will give us a complete proof system for SL. Later in the chapter, we introduce rules for quantifiers and identity.

Egyértelműen az ismétlési szabály nem engedi meg nekünk hogy bármi \emph{újat} mutassunk. Ahhoz több szabályra van szükségünk.
Ezen fejezet maradéka bevezető és elimináló szabályokat fog adni a kijelentéslogikai műveletekhez. Ez egy komplett bizonyíték rendszert ad nekünk KL-hez. Később a fejezetben bevezetünk szabályokat kvantorokhoz és az azonossághoz.

All of the rules introduced in this chapter are summarized starting on p.~\pageref{ProofRules}.

Mindegyik szabály, amelyet ebben a fejezetben vezetünk, be összegezve van a ~\pageref{ProofRules} oldalon.

\subsection*{Conjunction}
\subsection{Konjukció}

Think for a moment: What would you need to show in order to prove $E \eand F$?

Gondolkodj picit: Mit kellene megmutatnod ahhoz, hogy bizonyítsd $E \eand F$-et?

Of course, you could show $E \eand F$ by proving $E$ and separately proving $F$. 
This holds even if the two conjuncts are not atomic sentences. If you can prove $[(A \eor J) \eif V]$ and  $[(V \eif L) \eiff (F \eor N)]$, then you have effectively proven
$$[(A \eor J) \eif V] \eand [(V \eif L) \eiff (F \eor N)].$$
So this will be our conjunction introduction rule, which we abbreviate {\eand}I:

Persze, megmutathatnád $E \eand F$-et azáltal, hogy bizonyítod $E$-t és külön $F$-et.
Ez fennáll akkor is, ha a két konjukció nem atomi mondat. Ha be tudod bizonyítani, hogy  $[(A \eor J) \eif V]$ és  $[(V \eif L) \eiff (F \eor N)]$, akkor bebizonyítottad, hogy
$$[(A \eor J) \eif V] \eand [(V \eif L) \eiff (F \eor N)].$$
Tehát a konjukció bevezető szabályunk így fog kinézni rövidítve {\eand}I:
 
\begin{proof}
	\have[m]{a}{\script{A}}
	\have[n]{b}{\script{B}}
	\have[\ ]{c}{\script{A}\eand\script{B}} \ai{a, b}
\end{proof}

A line of proof must be justified by some rule, and here we have `{\eand}I m,n.' This means: Conjunction introduction applied to line $m$ and line $n$. These are variables, not real line numbers; $m$ is some line and $n$ is some other line. In an actual proof, the lines are numbered $1, 2, 3, \ldots$ and rules must be applied to specific line numbers. When we define the rule, however, we use variables to underscore the point that the rule may be applied to any two lines that are already in the proof. If you have $K$ on line 8 and $L$ on line 15, you can prove $(K\eand L)$ at some later point in the proof with the justification `{\eand}I 8, 15.'

Egy bizonyítás sorát mindig indokolni kell valamilyen szabállyal, és itt ez esetben „{\eand}I m,n.”
Ez a következőt jelenti: Konjukció bemutatás van alkalmazva $m$ és $n$ sorra. Ezek változók, nem valós sorszámok; m valamilyen sor és n valamilyen másik sor. Egy valós bizonyításban a sorok meg vannak számozva $1, 2, 3, \ldots$ formában és a szabályokat valamelyik specifikus sorszámhoz kell alkalmazni. Viszont, amikor definiáljuk a szabályt változókat használunk, hogy felhívjuk a figyelmet arra, hogy a szabályt bármelyik kettő, a bizonyításban már benne lévő sorra lehet alkalmazni. Ha $K$-d van a nyolcas soron és $L$ a tizenötösön, akkor be tudod bizonyítani $(K\eand L)$-t valamikor később a bizonyításban „{\eand}I 8, 15” indoklással.

Now, consider the elimination rule for conjunction. What are you entitled to conclude from a sentence like $E \eand F$? Surely, you are entitled to conclude $E$; if $E \eand F$ were true, then $E$ would be true. Similarly, you are entitled to conclude $F$. This will be our conjunction elimination rule, which we abbreviate {\eand}E:

Most vegyük figyelembe az elimináló szabályt a konjukcióhoz. Mire vagy jogosult következtetni egy olyan mondatból mint $E \eand F$?  Biztosan jogosultak vagyunk arra a következtetésre hogy $E$; ha $E \eand F$ igaz, akkor $E$ is igaz. Hasonlóan, jogosultak vagyunk arra a következtetésre hogy $F$. Ez lesz a konjukció elimináló szabályunk, amit rövidítünk úgy hogy {\eand}E:

\begin{proof}
	\have[m]{ab}{\script{A}\eand\script{B}}
	\have[\ ]{a}{\script{A}} \ae{ab}
	\have[\ ]{b}{\script{B}} \ae{ab}
\end{proof}

%GyRJ fordítása vége

%MRB fordítása kezdet

When you have a conjunction on some line of a proof, you can use {\eand}E to derive either of the conjuncts. The {\eand}E rule requires only one sentence, so we write one line number as the justification for applying it.

Ha konjunkció van egy bizonyítási sorban, használhatod {\eand}E-t, hogy levezesd bármelyik konjunktot. Az {\eand}E szabály csak egy mondatot igényel, ezért leírjuk az egyes sorszámot, hogy igazoljuk a használatát.

Even with just these two rules, we can provide some proofs. Consider this argument.

Csupán ezzel a két szabállyal már tudunk csinálni néhány bizonyítást.
Fontold meg ezt az érvelést.

\begin{earg}
\item[] $[(A\eor B)\eif(C\eor D)] \eand [(E \eor F) \eif (G\eor H)]$
\item[\therefore] $[(E \eor F) \eif (G\eor H)] \eand [(A\eor B)\eif(C\eor D)]$
\end{earg}
The main logical operator in both the premise and conclusion is conjunction. Since conjunction is symmetric, the argument is obviously valid. In order to provide a proof, we begin by writing down the premise. After the premises, we draw a horizontal line--- everything below this line must be justified by a rule of proof. So the beginning of the proof looks like this:

A fő logikai operátor a feltevésben és a következtetésben a konjunkció. Mivel a konjunkció szimmetrikus, az érvelés természetesen érvényes lesz. A bizonyítás biztosításának érdekében, elkezdjük leírni az előfeltételeket. Az előfeltételek után vízszintes vonalat rajzolunk -- minden, ami a vonal alatt van, a bizonyítási szabályok alapján igazolva kell legyen. Tehát a bizonyítás kezdete így néz ki:

\begin{proof}
	\hypo{ab}{{[}(A\eor B)\eif(C\eor D){]} \eand {[}(E \eor F) \eif (G\eor H){]}}
\end{proof}

From the premise, we can get each of the conjuncts by {\eand}E. The proof now looks like this:

A kezdőfeltételtől kezdve, mindegyik konjunkciót megkaphatjuk {\eand}E-vel. A bizonyítás most így néz ki: 

\begin{proof}
	\hypo{ab}{{[}(A\eor B)\eif(C\eor D){]} \eand {[}(E \eor F) \eif (G\eor H){]}}
	\have{a}{{[}(A\eor B)\eif(C\eor D){]}} \ae{ab}
	\have{b}{{[}(E \eor F) \eif (G\eor H){]}} \ae{ab}
\end{proof}

The rule {\eand}I requires that we have each of the conjuncts available somewhere in the proof. They can be separated from one another, and they can appear in any order. So by applying the {\eand}I rule to lines 3 and 2, we arrive at the desired conclusion. The finished proof looks like this:

Az {\eand}I szabályhoz szükséges, hogy az összes konjunkció elérhető legyen valahol a bizonyításban. Külön lehetnek egymástól és bármilyen sorrendben megjelenhetnek. Tehát, {\eand}I szabályt alkalmazva a 3-as és 2-es sorban, elérkezünk a kívánt következtetéshez. A befejezett bizonyítás így néz ki:

\begin{proof}
	\hypo{ab}{{[}(A\eor B)\eif(C\eor D){]} \eand {[}(E \eor F) \eif (G\eor H){]}}

	\have{a}{{[}(A\eor B)\eif(C\eor D){]}} \ae{ab}
	\have{b}{{[}(E \eor F) \eif (G\eor H){]}} \ae{ab}
	\have{ba}{{[}(E \eor F) \eif (G\eor H){]} \eand {[}(A\eor B)\eif(C\eor D){]}} \ai{b,a}
\end{proof}

This proof is trivial, but it shows how we can use rules of proof together to demonstrate the validity of an argument form. Also: Using a truth table to show that this argument is valid would have required a staggering 256 lines, since there are eight sentence letters in the argument.

Ez a bizonyítás triviális, de megmutatja, hogy hogyan tudjuk a bizonyítás szabályait együtt használni ahhoz, hogy igazoljunk egy érvelési formát. Továbbá látszik, hogy igazságtáblát használva  ennek az érvelésnek a levezetéséhez megdöbbentően sok, 256 sort kellett volna használni, hiszen nyolc mondatbetű szerepelt az érvelésben.

%When we defined a wff, we did not allow for conjunctions with more than two conjuncts. If we had done so, then we could define a more general version of the rules of proof for conjunction.

%MRB fordítása vége

%VB fordítása kezdet

\subsection*{Disjunction}
\subsection{Diszjunkció}
If $M$ were true, then $M \eor N$ would also be true. So the disjunction introduction rule ({\eor}I) allows us to derive a disjunction if we have one of the two disjuncts:

Ha $M$ igaz lenne, akkor az $M \eor N$ is igaz lenne. Szóval a diszjunkciót bemutató szabály ({\eor}I) megengedi, hogy származtassunk egy diszjunkciót ha rendelkezésünkre áll egy a két lenti esetből:

\begin{proof}
	\have[m]{a}{\script{A}}
	\have[\ ]{ab}{\script{A}\eor\script{B}}\oi{a}
	\have[\ ]{ba}{\script{B}\eor\script{A}}\oi{a}
\end{proof}

Notice that \script{B} can be \emph{any} sentence whatsoever. So the following is a legitimate proof:

Vegyük észre, hogy \script{B} \emph{bármilyen} kijelentés lehet. Szóval a következő egy igazolt bizonyítás:

\begin{proof}
	\hypo{m}{M}
	\have{mmm}{M \eor ([(A\eiff B) \eif (C \eand D)] \eiff [E \eand F])}\oi{m}
\end{proof}

It may seem odd that just by knowing $M$ we can derive a conclusion that includes sentences like $A$, $B$, and the rest--- sentences that have nothing to do with $M$. Yet the conclusion follows immediately by {\eor}I. This is as it should be: The truth conditions for the disjunction mean that, if \script{A} is true, then $\script{A}\eor \script{B}$ is true regardless of what \script{B} is. So the conclusion could not be false if the premise were true; the argument is valid.

Furcsának tűnhet, hogy csak azért mert ismerjük $M$-et olyan konklúzióra következtethetünk, melyek tartalmaznak dolgokat mint $A$, $B$, stb. -- kijelentéseket, amelyeknek semmi közük $M$-hez. Mégis, a konklúziót közvetlenül egy {\eor}I követi. Ez, ahogyan lennie kell: Az igazság feltételei a diszjunkcióhoz azt jelentik, hogy ha \script{A} igaz, akkor $\script{A}\eor \script{B}$ is igaz, függetlenül attól, hogy \script{B} milyen. Szóval a konklúzió nem lehet hamis ha a premissza igaz volt; az bizonyítás helyes.

Now consider the disjunction elimination rule. What can you conclude from $M \eor N$? You cannot conclude $M$. It might be $M$'s truth that makes $M \eor N$ true, as in the example above, but it might not. From $M \eor N$ alone, you cannot conclude anything about either $M$ or $N$ specifically. If you also knew that $N$ was false, however, then you would be able to conclude $M$.

Most gondoljuk át a diszjunkciós kiküszöbölés szabályt. Mit tudunk következtetni $M \eor N$-ből? Nem következtethetünk $M$-et. Lehetséges, hogy $M$ igazságértéke az ami igazzá teszi $M \eor N$-t, csak mint a fenti példában, de lehetséges, hogy nem. Egyedül $M \eor N$-ből, nem következtethetünk semmi specifikusra  $M$-ről vagy $N$-ről. Ha azonban tudtuk volna, hogy $N$ hamis, akkor mégis következtethettünk volna $M$-re.

This is just disjunctive syllogism, it will be the disjunction elimination rule ({\eor}E).

Ez nem más, mint a diszjunktív szillogizmus, ez lesz a diszjunkciós kiküszöbölési szabály. ({\eor}E).
\begin{multicols}{2}
\begin{proof}
	\have[m]{ab}{\script{A}\eor\script{B}}
	\have[n]{nb}{\enot\script{B}}
	\have[\ ]{a}{\script{A}} \oe{ab,nb}
\end{proof}

\begin{proof}
	\have[m]{ab}{\script{A}\eor\script{B}}
	\have[n]{na}{\enot\script{A}}
	\have[\ ]{b}{\script{B}} \oe{ab,nb}
\end{proof}

\end{multicols}

\subsection*{Conditional}
\subsection{Implikáció}

Consider this argument:

Gondoljuk át a következő érvelést:
\begin{earg}
\item[] $R \eor F$
\item[\therefore] $\enot R \eif F$
\end{earg}
The argument is certainly a valid one. What should the conditional introduction rule be, such that we can draw this conclusion?

Az érvelés bizonyosan igaz. Milyennek kellene lennie az implikáció bevezető szabályának, hogy erre a következtetésre juthassunk?

%VB fordítása vége

%KN fordítása kezdet 

We begin the proof by writing down the premise of the argument and drawing a horizontal line, like this:

A bizonyítást úgy kezdjük, hogy leírjuk a kijelentés premisszáját, majd egy vízszintes vonalat húzunk:

\begin{proof}
	\hypo{rf}{R \eor F}
\end{proof}

If we had $\enot R$ as a further premise, we could derive $F$ by the {\eor}E rule. We do not have $\enot R$ as a premise of this argument, nor can we derive it directly from the premise we do have--- so we cannot simply prove $F$. What we will do instead is start a \emph{subproof}, a proof within the main proof. When we start a subproof, we draw another vertical line to indicate that we are no longer in the main proof. Then we write in an assumption for the subproof. This can be anything we want. Here, it will be helpful to assume $\enot R$. Our proof now looks like this:

Ha $\enot R$ további premissza volna, $F$-et származtathatnánk a {\eor}E szabályból. $\enot R$ most nem a premisszája a kijelentésünknek, és a meglévőből sem tudjuk származtatni--- így nem tudjuk egyszerűen bizonyítani $F$-et. Ehelyett egy \emph{albizonyítást} kezdünk, azaz egy bizonyítást a fő bizonyításon belül. Amikor egy albizonyítást kezdünk, függőleges vonalat húzunk, ezzel elválasztva a fő bizonyítástól. Ezután beírunk egy feltételt az albizonyításba. Ez bármi lehet amit csak szeretnénk. Esetünkben hasznos lehet a $\enot R$ feltétel. Jelenleg így néz ki tehát a bizonyítás:

\begin{proof}
	\hypo{rf}{R \eor F}
	\open
		\hypo{nr}{\enot R}
	\close
\end{proof}

It is important to notice that we are not claiming to have proven $\enot R$. We do not need to write in any justification for the assumption line of a subproof. You can think of the subproof as posing the question: What could we show \emph{if} $\enot R$ were true? For one thing, we can derive $F$. So we do:

Fontos megjegyezni, hogy nem állítjuk $\enot R$ bebizonyítását. Nem szükséges igazolnunk az albizonyítás feltételezésének a sorát. Gondolhatunk az albizonyításra, mint a következő kérdésre: Mit tudnánk megmutatni, \emph{ha} $\enot R$ igaz volna? Elsőkörön származtathatnánk $F$-t. Tegyük is meg:

\begin{proof}
	\hypo{rf}{R \eor F}
	\open
		\hypo{nr}{\enot R}
		\have{f}{F}\oe{rf, nr}
	\close
\end{proof}

This has shown that \emph{if} we had $\enot R$ as a premise, \emph{then} we could prove $F$. In effect, we have proven $\enot R \eif F$. So the conditional introduction rule ({\eif}I) will allow us to close the subproof and derive $\enot R \eif F$ in the main proof. Our final proof looks like this:

Ez megmutatta, hogy \emph{ha} $\enot R$ egy premissza volna, \emph{akkor} bizonyítani tudnánk $F$-et. Cserébe bebizonyítottuk ezt: $\enot R \eif F$. Tehát az implikáció bevezetés szabálya ({\eif}I) megengedi számunkra az albizonyítás lezárását, és $\enot R \eif F$ származtatását a fő bizonyításban. A végső bizonyítás tehát:


\begin{proof}
	\hypo{rf}{R \eor F}
	\open
		\hypo{nr}{\enot R}
		\have{f}{F}\oe{rf, nr}
	\close
	\have{nrf}{\enot R \eif F}\ci{nr-f}
\end{proof}


Notice that the justification for applying the {\eif}I rule is the entire subproof. Usually that will be more than just two lines.

Vegyük észre, hogy a {\eif}I szabály használatát a teljes albizonyítás tette lehetővé. Általában ez több, mint két sor lesz.

It may seem as if the ability to assume anything at all in a subproof would lead to chaos: Does it allow you to prove any conclusion from any premises? The answer is no, it does not. Consider this proof:

Úgy tűnhet, hogy a bármi feltételezésének a lehetősége az albizonyításban káoszhoz vezethet: Bármilyen következtetést bizonyíthatunk bármilyen premisszából? A válasz nem. Nézzük ezt a bizonyítást:

%KN fordítása vége

%KF fordítása kezdet

\begin{proof}
	\hypo{a}{\script{A}}
	\open
		\hypo{b1}{\script{B}}
		\have{b2}{\script{B}} \by{R}{b1}
	\close
\end{proof}

It may seem as if this is a proof that you can derive any conclusions \script{B} from any premise \script{A}. When the vertical line for the subproof ends, the subproof is \emph{closed}. In order to complete a proof, you must close all of the subproofs. And you cannot close the subproof and use the R rule again on line 4 to derive \script{B} in the main proof. Once you close a subproof, you cannot refer back to individual lines inside it.

Úgy tűnhet, hogy ez egy olyan bizonyítás, amellyel bármely \script{B} következtetést bármely \script{A} premisszából származtathatunk. Mikor a vízszintes vonal a végéhez ér, az albizonyítás \emph{lezárul}. Annak érdekében, hogy a bizonyítást teljesíthessük minden albizonyítást le kell zárnunk. Nem zárhatjuk le az albizonyítást és használhatjuk az R szabályt újra a 4. sorban, hogy \script{B}-t származtathassuk a fő bizonyításban. Amennyiben már lezártunk egy albizonyítást, többé nem hivatkozhatunk vissza annak egyes soraira.

Closing a subproof is called \emph{discharging} the assumptions of that subproof. So we can put the point this way: You cannot complete a proof until you have discharged all of the assumptions besides the original premises of the argument.

Egy albizonyítás lezárását az albizonyítás feltételezéseinek \emph{teljesítésének} hívjuk. A lényeget összefoglalva: Nem teljesíthetünk egy bizonyítást, amíg nem teljesítettük az összes feltételezést a kijelentésünk eredeti premisszája mellet.

Of course, it is legitimate to do this:

Természetesen az  elfogadott, ha ezt csináljuk:

\begin{proof}
	\hypo{a}{\script{A}}
	\open
		\hypo{b1}{\script{B}}
		\have{b2}{\script{B}} \by{R}{b1}
	\close
	\have{bb}{\script{B}\eif\script{B}} \ci{b1-b2}
\end{proof}

This should not seem so strange, though. Since \script{B}\eif\script{B} is a tautology, no particular premises should be required to validly derive it. (Indeed, as we will see, a tautology follows from any premises.)

Ennek nem kell ilyen különösnek lennie. Mivel \script{B}\eif\script{B} egy tautológia, ezért nincs szükség adott premisszára, hogy érvényesen származtathassuk. (És valóban, ahogy azt látni fogjuk, egy tautológia bármely premisszából következik.)

Put in a general form, the {\eif}I rule looks like this:

Általános formával írva a {\eif}I szabály így fog kinézni:

\begin{proof}
	\open
		\hypo[m]{a}{\script{A}} \by{want \script{B}}{}
		\have[n]{b}{\script{B}}
	\close
	\have[\ ]{ab}{\script{A}\eif\script{B}}\ci{a-b}
\end{proof}

\begin{proof}
	\open
		\hypo[m]{a}{\script{A}} \by{\script{B}-t akarjuk}{}
		\have[n]{b}{\script{B}}
	\close
	\have[\ ]{ab}{\script{A}\eif\script{B}}\ci{a-b}
\end{proof}

When we introduce a subproof, we typically write what we want to derive in the column. This is just so that we do not forget why we started the subproof if it goes on for five or ten lines. There is no `want' rule. It is a note to ourselves and not formally part of the proof.

Amikor bevezetünk egy albizonyítást, akkor általában azt írjuk az oszlopba, amit származtatni akarunk. Mindezt azért, hogy ne felejtsük el, hogy miért is indítottuk az albizonyítást, akár 5 vagy 10 sorral korábban. Nincs arra vonatkozó szabály, hogy mit "akarunk". Ez csak egy megjegyzés magunknak és nem formális része a bizonyításnak.

Although it is always permissible to open a subproof with any assumption you please, there is some strategy involved in picking a useful assumption. Starting a subproof with an arbitrary, wacky assumption would just waste lines of the proof. In order to derive a conditional by the {\eif}I, for instance, you must assume the antecedent of the conditional in a subproof.

Habár bármikor megengedett egy albizonyítás nyitása bármilyen kedvünkre való feltételezéssel, van némi stratégia a számunkra hasznos feltételezés kiválasztásában. Egy albizonyítás elkezdése tetszőleges, kiszámíthatatlan feltételezésekkel csak a bizonyítás sorait pazarolja. Annak érdekében, hogy származtassunk egy {\eif}I implikációt például, feltételeznünk kell az implikáció előzményét egy albizonyításban.

%KF fordítása vége

%DD fordítása kezdet

The {\eif}I rule also requires that the consequent of the conditional be the last line of the subproof. It is always permissible to close a subproof and discharge its assumptions, but it will not be helpful to do so until you get what you want.

Az {\eif}I szabály azt is megköveteli, hogy az inplikáció következménye legyen az alsó réteg utolsó sora. Mindig megengedett az alsó korlát lezárása és a feltételezéseinek teljesítése, de addig nem lesz hasznos, amíg meg nem kapja azt, amit szeretne.

Now consider the conditional elimination rule. Nothing follows from $M\eif N$ alone, but if we have both $M \eif N$ and $M$, then we can conclude $N$. This rule, modus ponens, will be the conditional elimination rule ({\eif}E).

Most vegyük az implikáció eliminálási szabályát. Semmi nem következik az $M\eif N$-ből önmagában, de ha $M \eif N$ és $M$ egyaránt szerepel, akkor következtethetünk $N$-t. Ez a modus ponens szabály lesz a feltételes eliminációs szabály ({\eif}E).

\begin{proof}
	\have[m]{ab}{\script{A}\eif\script{B}}
	\have[n]{a}{\script{A}}
	\have[\ ]{b}{\script{B}} \ce{ab,a}
\end{proof}

Now that we have rules for the conditional, consider this argument:

Most, hogy megvannak a feltételekre vonatkozó szabályok, nézzük meg az alábbi állításokat:

\label{proofHS}
\begin{earg}
\item[] $P \eif Q$
\item[] $Q \eif R$
\item[\therefore] $P \eif R$
\end{earg}
We begin the proof by writing the two premises as assumptions. Since the main logical operator in the conclusion is a conditional, we can expect to use the {\eif}I rule. For that, we need a subproof--- so we write in the antecedent of the conditional as assumption of a subproof:

A bizonyítást azzal kezdjük, hogy feltesszük a két állítást. Mivel a következtetés fő logikai operátora az implikáció, elvárható az {\eif}I szabály használata. Ehhez szükség van egy alsó védelemre -- tehát azt írjuk az állítás alá, mint egy alsó védelem feltétel:

\begin{proof}
	\hypo{pq}{P \eif Q}
	\hypo{qr}{Q \eif R}
	\open
		\hypo{p}{P}
	\close
\end{proof}

We made $P$ available by assuming it in a subproof, allowing us to use {\eif}E on the first premise. This gives us $Q$, which allows us to use {\eif}E on the second premise. Having derived  $R$, we close the subproof. By assuming $P$ we were able to prove $R$, so we apply the {\eif}I rule and finish the proof.

A $P$-t úgy tettük elérhetővé, hogy azt egy alsó védelemben feltételezzük, lehetővé téve az {\eif}E használatát az első feltevésen. Ez megadja a $Q$ értéket, amely lehetővé teszi az {\eif}E használatát a második feltevésen. Miután megkaptuk $R$-t, lezárjuk az alsó réteget. $P$ feltételezésével képesek voltunk $R$-t bizonyítani, tehát az {\eif}I szabályt alkalmazzuk és befejezzük a bizonyítást.

\label{HSproof}
\begin{proof}
	\hypo{pq}{P \eif Q}
	\hypo{qr}{Q \eif R}
	\open
		\hypo{p}{P}\by{want $R$}{}
		\have{q}{Q}\ce{pq,p}
		\have{r}{R}\ce{qr,q}
	\close
	\have{pr}{P \eif R}\ci{p-r}
\end{proof}

\label{HSproof}
\begin{proof}
	\hypo{pq}{P \eif Q}
	\hypo{qr}{Q \eif R}
	\open
		\hypo{p}{P}\by{$R$-t szeretnénk}{}
		\have{q}{Q}\ce{pq,p}
		\have{r}{R}\ce{qr,q}
	\close
	\have{pr}{P \eif R}\ci{p-r}
\end{proof}

%DD fordítása vége

%KLP fordítása kezdet

\subsection*{Biconditional}
\subsection{Ekvivalencia}
The rules for the biconditional will be like double-barreled versions of the rules for the conditional.
Az ekvivalencia szabályai olyanok lesznek, mint az implikáció kétszeres változatai.

In order to derive $W \eiff X$, for instance, you must be able to prove $X$ by assuming $W$ \emph{and} prove $W$ by assuming $X$. The biconditional introduction rule ({\eiff}I) requires two subproofs. The subproofs can come in any order, and the second subproof does not need to come immediately after the first--- but schematically, the rule works like this:

Például a $W \eiff X$ származtatáshoz be kell bizonyítani $X$-et $W$ feltételezésével és $W$-t $X$ feltételezésével. Az ekvivalencia bevezetési szabály ({\eiff}I) két részbizonyítást követel. A részbizonyítások bármilyen sorrendben elvégezhetők és a második szabálynak nem kell közvetlenül az első után következnie. Vázlatosan a szabály így működik:

\begin{proof}
	\open
		\hypo[m]{a1}{\script{A}} \by{want \script{B}}{}
		\have[n]{b1}{\script{B}}
	\close
	\open
		\hypo[p]{b2}{\script{B}} \by{want \script{A}}{}
		\have[q]{a2}{\script{A}}
	\close
	\have[\ ]{ab}{\script{A}\eiff\script{B}}\bi{a1-b1,b2-a2}
\end{proof}

\begin{proof}
	\open
		\hypo[m]{a1}{\script{A}} \by{\script{B}-t akarjuk}{}
		\have[n]{b1}{\script{B}}
	\close
	\open
		\hypo[p]{b2}{\script{B}} \by{\script{A}-t akarjuk}{}
		\have[q]{a2}{\script{A}}
	\close
	\have[\ ]{ab}{\script{A}\eiff\script{B}}\bi{a1-b1,b2-a2}
\end{proof}

The biconditional elimination rule ({\eiff}E) lets you do a bit more than the conditional rule. If you have the left-hand subsentence of the biconditional, you can derive the right-hand subsentence. If you have the right-hand subsentence, you can derive the left-hand subsentence. This is the rule:

Az ekvivalencia eliminálási szabály ({\eiff}E) többet tesz lehetővé, mint az implikációs szabály. Ha az ekvivalencia bal oldali része megvan, abból levezethető a jobb oldali rész. Ha a jobb oldali rész van meg, a bal oldal vezethető le. Ez a szabály:

\begin{multicols}{2}
\begin{proof}
	\have[m]{ab}{\script{A}\eiff\script{B}}
	\have[n]{a}{\script{A}}
	\have[\ ]{b}{\script{B}} \be{ab,a}
\end{proof}

\begin{proof}
	\have[m]{ab}{\script{A}\eiff\script{B}}
	\have[n]{a}{\script{B}}
	\have[\ ]{b}{\script{A}} \be{ab,a}
\end{proof}
\end{multicols}




\subsection*{Negation}
\subsection{Negáció}
Here is a simple mathematical argument in English:
\begin{earg}
\item[] Assume there is some greatest natural number. Call it $A$.
\item[] That number plus one is also a natural number.
\item[] Obviously, $A+1 > A$.
\item[] So there is a natural number greater than $A$.
\item[] This is impossible, since $A$ is assumed to be the greatest natural number.
\item[\therefore] There is no greatest natural number.
\end{earg}

Íme egy egyszerű matematikai érv:
\begin{earg}
\item[] Tegyük fel, hogy létezik legnagyobb természetes szám. Hívjuk $A$-nak.
\item[] Ehhez a számhoz egyet hozzáadva is természetes számot kapunk.
\item[] Egyértlemű, hogy $A+1 > A$.
\item[] Tehát létezik egy $A$-nál nagyobb természetes szám.
\item[] Ez lehetetlen, hiszen azt feltételeztük, hogy $A$ a legnagyobb természetes szám.
\item[\therefore] Nem létezik legnagyobb természetes szám.
\end{earg}

This argument form is traditionally called a \emph{reductio}. Its full Latin name is \emph{reductio ad absurdum}, which means `reduction to absurdity.' In a reductio, we assume something for the sake of argument--- for example, that there is a greatest natural number. Then we show that the assumption leads to two contradictory sentences--- for example, that $A$ is the greatest natural number and that it is not. In this way, we show that the original assumption must have been false.

Ezt az érvelési formát hagyományosan \emph{reductio}-nak nevezzük. A teljes latin neve \emph{reductio ad absurdum}, ami azt jelenti „visszavezetés az abszurdra.” Reductio esetén az érvek kedvéért feltételezzünk valamit - például, hogy létezik egy legnagyobb természetes szám. Aztán megmutatjuk, hogy a feltételezés két ellentmondásos mondathoz vezet - például, hogy $A$ a legnagyobb természetes szám, és hogy nem. Ily módon megmutatjuk, hogy az eredeti feltételezés hamis volt.
%KLP fordítása vége

%MK fordítása kezdet

The basic rules for negation will allow for arguments like this. If we assume something and show that it leads to contradictory sentences, then we have proven the negation of the assumption. This is the negation introduction ({\enot}I) rule:

A negáció(tagadás) alapvető szabályai megengedik az ilyen érvelést. Ha állítunk valamit és megmutatjuk, hogy ellentmondásos mondatokhoz vezet, akkor bizonyítottuk a feltevés negációját(tagadását). Ez a negáció bevezetésének ({\enot}I) szabálya:

\begin{proof}
\open
	\hypo[m]{na}{\script{A}}\by{for reductio}{}
	\have[n]{b}{\script{B}}
	\have{nb}{\enot\script{B}}
\close
\have{a}[\ ]{\enot\script{A}}\ni{na-nb}
\end{proof}

\begin{proof}
\open
	\hypo[m]{na}{\script{A}}\by{reductio ad absurdum céljából}{}
	\have[n]{b}{\script{B}}
	\have{nb}{\enot\script{B}}
\close
\have{a}[\ ]{\enot\script{A}}\ni{na-nb}
\end{proof}

For the rule to apply, the last two lines of the subproof must be an explicit contradiction: some sentence followed on the next line by its negation. We write `for reductio' as a note to ourselves, a reminder of why we started the subproof. It is not formally part of the proof, and you can leave it out if you find it distracting.

Ahhoz, hogy alkalmazhassuk a szabályt, a részbizonyítás utolsó két sorának egyértelmű ellentmondásnak kell lennie: egy mondatot a következő sorban a tagadása követ. Jegyzetként odaírjuk magunknak, hogy „reductio ad absurdum céljából”, vagy „ellentmondás céljából”, azért, hogy emlékeztessen, hogy miért kezdtük el a részbizonyítást. Normál esetben ez nem része a bizonyításnak, ki is hagyhatod, ha zavarónak találod.

To see how the rule works, suppose we want to prove the law of non-contradiction: $\enot(G \eand \enot G)$. We can prove this without any premises by immediately starting a subproof. We want to apply {\enot}I to the subproof, so we assume $(G \eand \enot G)$. We then get an explicit contradiction by {\eand}E. The proof looks like this:

Hogy lássuk, hogy működik a szabály, tegyük fel, hogy bizonyítani akarjuk a nem-ellentmondás szabályát: $\enot(G \eand \enot G)$. Ezt be tudjuk bizonyítani premisszák nélkül, egyből részbizonyítással. Alkalmazni akarjuk {\enot}I-t a részbizonyításra, így feltesszük, hogy $(G \eand \enot G)$. Egyértelmű ellentmondásra jutunk {\eand}E által. A bizonyítás így néz ki:

\begin{proof}
	\open
		\hypo{gng}{G\eand \enot G}\by{for reductio}{}
		\have{g}{G}\ae{gng}
		\have{ng}{\enot G}\ae{gng}
	\close
	\have{ngng}{\enot(G \eand \enot G)}\ni{gng-ng}
\end{proof}

\begin{proof}
	\open
		\hypo{gng}{G\eand \enot G}\by{ellentmondás céljából}{}
		\have{g}{G}\ae{gng}
		\have{ng}{\enot G}\ae{gng}
	\close
	\have{ngng}{\enot(G \eand \enot G)}\ni{gng-ng}
\end{proof}

The {\enot}E rule will work in much the same way. If we assume \enot\script{A} and show that it leads to a contradiction, we have effectively proven \script{A}. So the rule looks like this:

A {\enot}E szabály körülbelül ugyanígy fog működni. Ha feltesszük, hogy \enot\script{A} és megmutatjuk, hogy ellentmondáshoz vezet, hatékonyan bizonyítottuk \script{A}-t. Így a szabály a következőképp néz ki:

\begin{proof}
\open
	\hypo[m]{na}{\enot\script{A}}\by{for reductio}{}
	\have[n]{b}{\script{B}}
	\have{nb}{\enot\script{B}}
\close
\have{a}[\ ]{\script{A}}\ne{na-nb}
\end{proof}

\begin{proof}
\open
	\hypo[m]{na}{\enot\script{A}}\by{ellentmondás céljából}{}
	\have[n]{b}{\script{B}}
	\have{nb}{\enot\script{B}}
\close
\have{a}[\ ]{\script{A}}\ne{na-nb}
\end{proof}

\section*{Derived rules}
\section{Származtatott szabályok}
The rules of the natural deduction system are meant to be systematic. There is an introduction and an elimination rule for each logical operator, but why these basic rules rather than some others? Many natural deduction systems have a disjunction elimination rule that works like this:

A természetes levezetési(deduktív) rendszer szabályainak szisztematikusnak kell lenniük. Minden logikai operátorhoz van egy bevezető és egy kiküszöbölő(elimináló) szabály, de miért ilyen alapvető szabályok vannak inkább mint valamilyen másfélék? Sok természetes levezetési rendszerben van egy szétválasztó(diszjunktív) kiküszöbölési szabály, ami így működik:

\begin{proof}
	\have[m]{ab}{\script{A}\eor\script{B}}
	\have[n]{ac}{\script{A}\eif\script{C}}
	\have[o]{bc}{\script{B}\eif\script{C}}
	\have[\ ]{c}{\script{C}} \by{DIL}{ab,ac,bc}
\end{proof}

Let's call this rule Dilemma (DIL) It might seem as if there will be some proofs that we cannot do with our proof system, because we do not have this as a basic rule. Yet this is not the case. Any proof that you can do using the Dilemma rule can be done with basic rules of our natural deduction system. Consider this proof:

Nevezzük ezt a szabályt Dilemmának (DIL). Úgy tűnhet, mintha lenne néhány bizonyítás, amit nem tudunk megcsinálni a bizonyítási rendszerünkkel, mivel ez nincs az alapvető szabályaink között. Mégsem ez a helyzet. Bármely bizonyítás, amit meg tudsz csinálni a Dilemma szabály használatával, megcsinálható a természetes levezetési rendszerünk alapvető szabályaival. Vegyük a következő bizonyítást:

\begin{proof}
	\hypo{ab}{\script{A}\eor\script{B}}
	\hypo{ac}{\script{A}\eif\script{C}}
	\hypo{bc}{\script{B}\eif\script{C}}\by{want \script{C}}{}
	\open
		\hypo{nc}{\enot \script{C}}\by{for reductio}{}
		\open
			\hypo{a1}{\script{A}}\by{for reductio}{}
			\have{c1}{\script{C}}\ce{ac, a1}
			\have{nc1}{\enot\script{C}}\by{R}{nc}
		\close
		\have{na}{\enot\script{A}}\ni{a1-nc1}
		\open
			\hypo{b2}{\script{B}}\by{for reductio}{}
			\have{c2}{\script{C}}\ce{bc, b2}
			\have{nc2}{\enot\script{C}}\by{R}{nc}
		\close
		\have{b}{\script{B}}\oe{ab, na}
		\have{nb}{\enot\script{B}}\ni{b2-nc2}
	\close
	\have{c}{\script{C}} \ne{nc-nb}
\end{proof}

\begin{proof}
	\hypo{ab}{\script{A}\eor\script{B}}
	\hypo{ac}{\script{A}\eif\script{C}}
	\hypo{bc}{\script{B}\eif\script{C}}\by{ \script{C}-t akarunk}{}
	\open
		\hypo{nc}{\enot \script{C}}\by{ellentmondás céljából}{}
		\open
			\hypo{a1}{\script{A}}\by{ellentmondás céljából}{}
			\have{c1}{\script{C}}\ce{ac, a1}
			\have{nc1}{\enot\script{C}}\by{R}{nc}
		\close
		\have{na}{\enot\script{A}}\ni{a1-nc1}
		\open
			\hypo{b2}{\script{B}}\by{ellentmondás céljából}{}
			\have{c2}{\script{C}}\ce{bc, b2}
			\have{nc2}{\enot\script{C}}\by{R}{nc}
		\close
		\have{b}{\script{B}}\oe{ab, na}
		\have{nb}{\enot\script{B}}\ni{b2-nc2}
	\close
	\have{c}{\script{C}} \ne{nc-nb}
\end{proof}

%MK fordítása vége

%HT fordítása kezdet

\script{A}, \script{B}, and \script{C} are meta-variables. They are not symbols of SL, but stand-ins for arbitrary sentences of SL. So this is not, strictly speaking, a proof in SL. It is more like a recipe. It provides a pattern that can prove anything that the Dilemma rule can prove, using only the basic rules of SL. This means that the Dilemma rule is not really necessary. Adding it to the list of basic rules would not allow us to derive anything that we could not derive without it.

\script{A}, \script{B}, és \script{C} metaváltozók. Ezek nem KL szimbólumai, hanem KL tetszőleges mondatainak helyettesítései. Vagyis ez tulajdonképpen nem egy bizonyítás KL-ben. Inkább olyan, mint egy recept. Ad egy mintát, ami tud bizonyítani mindent amit a Dilemma szabály is tud, KL alapszabályait felhasználva. Ez azt jelenti, hogyy a Dilemma szabály igazán nem is szükséges. Hozzáadva az alap szabályok listájához nem engedné meg nekünk azt, hogy származtassunk akármit amit nem tudnánk származtathatni nélküle.

Nevertheless, the Dilemma rule would be convenient. It would allow us to do in one line what requires eleven lines and several nested subproofs with the basic rules. So we will add it to the proof system as a derived rule.

Azonban a Dilemma szabály kényelmes lenne. Megengedné nekünk hogy egy sorból megvalósítsuk azt amit tizenegy sor és számos beágyazott albizonyítás igényel az alap szabályokkal. Vagyis hozzáadjuk a bizonyítási rendszerhez mint származtatott szabály.

A \define{derived rule} is a rule of proof that does not make any new proofs possible. Anything that can be proven with a derived rule can be proven without it. You can think of a short proof using a derived rule as shorthand for a longer proof that uses only the basic rules. Anytime you use the Dilemma rule, you could always take ten extra lines and prove the same thing without it.

Egy \define{származtatott szabály} egy bizonyítási szabály ami nem tesz lehetővé új bizonyításokat. Akármi ami bizonyítható származtatott szabállyal, az bizonyítható anélkül is. Gondolhatunk egy rövid bizonyításra, ami származtatott szabályt használ, mint rövidítése egy hosszab bizonyításnak, ami csak az alapvető szabályokat használja. Akármikor, amikor a Dilemma szabályt használod, mindig vehetsz 10 extra sort és bizonyíthatod anélkül is.

For the sake of convenience, we will add several other derived rules. One is \emph{modus tollens} (MT).

A kényelmesség kedvéért, hozzáadunk több származtatott szabályt. Egyik a \emph{modus tollens} (MT).

\begin{proof}
	\have[m]{ab}{\script{A}\eif\script{B}}
	\have[n]{a}{\enot\script{B}}
	\have[\ ]{b}{\enot\script{A}} \by{MT}{ab,a}
\end{proof}

We leave the proof of this rule as an exercise. Note that if we had already proven the MT rule, then the proof of the DIL rule could have been done in only five lines.

Ennek a szabálynak a bizonyítását gyakorlásnak hagyjuk. Jegyezzük meg, ha már bebizonyítottuk az MT szabályt, akkor a DIL szabály bizonyítása megoldható csupán öt sorból.

We also add hypothetical syllogism (HS) as a derived rule. We have already given a proof of it on p.~\pageref{HSproof}.

Hozzáadjuk a láncszabályt is (hipotetikus szillogizmus, HS), mint származtatott szabály. Már van hozzá egy megadott bizonyításunk a \pageref{HSproof}. oldalon.

\begin{proof}
	\have[m]{ab}{\script{A}\eif\script{B}}
	\have[n]{bc}{\script{B}\eif\script{C}}
	\have[\ ]{ac}{\script{A}\eif\script{C}}\by{HS}{ab,bc}
\end{proof}


\section*{Rules of replacement}
\section{Helyettesítési szabályok}
%\nix{could be beefier}
Consider how you would prove this argument: $F\eif(G\eand H)$, \therefore\ $F\eif G$

%\nix{lehetne nagyobb}
Gondoljunk bele, hogyan tudnánk bebizonyítani ezt az állítást: $F\eif(G\eand H)$, \therefore\ $F\eif G$

Perhaps it is tempting to write down the premise and apply the {\eand}E rule to the conjunction $(G \eand H)$. This is impermissible, however, because the basic rules of proof can only be applied to whole sentences. We need to get $(G \eand H)$ on a line by itself. We can prove the argument in this way:

Talán csábító lehet leírni a feltételt és vonatkoztatni a {\eand}E szabályt a $(G \eand H)$ kapcsolathoz. Ez azonban megengedhetetlen, mert a bizonyítás alap szabályait csak egész momdatokra lehet alkalmazni. Meg kell kapnunk $(G \eand H)$-t egy saját sorba. Így tudjuk ezt az érvet bizonyítani:

%HT fordítása vége

%MB fordítása kezdet

\begin{proof}
	\hypo{fgh}{F\eif(G\eand H)}
	\open
		\hypo{f}{F}\by{want $G$}{}
		\have{gh}{G \eand H}\ce{fgh,f}
		\have{g}{G}\ae{gh}
	\close
	\have{fg}{F \eif G}\ci{f-g}
\end{proof}

We will now introduce some derived rules that may be applied to part of a sentence. These are called \define{rules of replacement}, because they can be used to replace part of a sentence with a logically equivalent expression. One simple rule of replacement is commutivity (abbreviated Comm), which says that we can swap the order of conjuncts in a conjunction or the order of disjuncts in a disjunction. We define the rule this way:

Most be fogunk mutatni néhány deriválási szabályt, amiket alkalmazhatunk a következőkben. Ezeket csereszabályoknak nevezik, mert alkalmazhatóak mondatok egy részének helyettesítésére egy velük logikailag egyenértékű kifejezéssel. Egy egyszerű helyettesítési szabály a kommutivitás (rövidítése: Comm), ami kimondja, hogy kicserélhetjük a konjunktumok sorrendjét egy konjunkcióban vagy a diszjunktumok sorrendjét egy diszjunkcióban.  Így definiáljuk a szabályt:

\begin{center}
\begin{tabular}{rl}
$(\script{A}\eand\script{B}) \Longleftrightarrow (\script{B}\eand\script{A})$\\
$(\script{A}\eor\script{B}) \Longleftrightarrow (\script{B}\eor\script{A})$\\
$(\script{A}\eiff\script{B}) \Longleftrightarrow (\script{B}\eiff\script{A})$
& Comm
\end{tabular}
\end{center}

The bold arrow means that you can take a subformula on one side of the arrow and replace it with the subformula on the other side. The arrow is double-headed because rules of replacement work in both directions.

A félkövér nyíl azt jelenti, hogy a két oldalán lévő részformulák felcserélhetőek egymással. A nyíl mind a két irányba mutat, mert a csereszabályok mind a két irányba működnek. 

Consider this argument: $(M \eor P) \eif (P \eand M)$, \therefore\ $(P \eor M) \eif (M \eand P)$

Vegyük ezt az érvet: $(M \eor P) \eif (P \eand M)$, \therefore\ $(P \eor M) \eif (M \eand P)$

It is possible to give a proof of this using only the basic rules, but it will be long and inconvenient. With the Comm rule, we can provide a proof easily:

Ezt lehet igazolni csak az alapszabályok felhasználásával is, de így elég hosszú és kellemetlen lenne. A Comm szabály segítségével könnyen bizonyíthatjuk: 

\begin{proof}
	\hypo{1}{(M \eor P) \eif (P \eand M)}
	\have{2}{(P \eor M) \eif (P \eand M)}\by{Comm}{1}
	\have{n}{(P \eor M) \eif (M \eand P)}\by{Comm}{2}
\end{proof}

Another rule of replacement is double negation (DN). With the DN rule, you can remove or insert a pair of negations anywhere in a sentence. This is the rule:

A helyettesítés egy másik szabálya a kettős tagadás (DN). A DN szabály segítségével eltávolíthatsz vagy beszúrhatsz egy pár negációt bárhova egy kifejezésben. Ez a szabály: 

\begin{center}
\begin{tabular}{rl}
$\enot\enot\script{A} \Longleftrightarrow \script{A}$ & DN
\end{tabular}
\end{center}

Two more replacement rules are called De Morgan's Laws, named for the 19th-century British logician August De Morgan. (Although De Morgan did discover these laws, he was not the first to do so.) The rules capture useful relations between negation, conjunction, and disjunction. Here are the rules, which we abbreviate DeM:

Két további helyettesítő szabályt hívnak a De Morgan törvényeknek (azonosságoknak), amelyeket a 19. századi brit logikusról, August De Morgan-ről neveztek el. (Bár De Morgan felfedezte ezeket a törvényeket, mégsem ő volt az első.) A szabályok hasznos kapcsolatokat rögzítenek a tagadás, a konjunkció és a diszjunkció közt. Itt vannak a szabályok, amiket mi csak DeM-nek rövidítünk:

%MB fordítása vége

%LLM fordítása kezdet

\begin{center}
\begin{tabular}{rl}
$\enot(\script{A}\eor\script{B}) \Longleftrightarrow (\enot\script{A}\eand\enot\script{B})$\\
$\enot(\script{A}\eand\script{B}) \Longleftrightarrow (\enot\script{A}\eor\enot\script{B})$
& DeM
\end{tabular}
\end{center}

Because $\script{A}\eif\script{B}$ is a \emph{material conditional}, it is equivalent to $\enot\script{A}\eor\script{B}$. A further replacement rule captures this equivalence. We abbreviate the rule MC, for `material conditional.' It takes two forms:

Mivel $\script{A}\eif\script{B}$ is \emph{anyagi feltétel}, egyenértékű $\enot\script{A}\eor\script{B}$-vel. Egy további helyettesítő szabály rögzíti ezt az egyenértékűséget. Az MC szabályt rövidítjük az „anyagi feltétel” kifejezésre. Ez kétféle lehet:

\begin{center}
\begin{tabular}{rl}
$(\script{A}\eif\script{B}) \Longleftrightarrow (\enot\script{A}\eor\script{B})$ &\\
$(\script{A}\eor\script{B}) \Longleftrightarrow (\enot\script{A}\eif\script{B})$ & MC
\end{tabular}
\end{center}

Now consider this argument: $\enot(P \eif Q)$, \therefore\ $P \eand \enot Q$

Most vegyük ezt az állítást: $\enot(P \eif Q)$, \therefore\ $P \eand \enot Q$

As always, we could prove this argument using only the basic rules. With rules of replacement, though, the proof is much simpler:

Mint mindig, ezt az állítást csak az alapszabályok segítségével be tudnánk bizonyítani. A csereszabályokkal azonban a bizonyítás sokkal egyszerűbb:

\begin{proof}
	\hypo{1}{\enot(P \eif Q)}
	\have{2}{\enot(\enot P \eor Q)}\by{MC}{1}
	\have{3}{\enot\enot P \eand \enot Q}\by{DeM}{2}
	\have{4}{P \eand \enot Q}\by{DN}{3}
\end{proof}

A final replacement rule captures the relation between conditionals and biconditionals. We will call this rule biconditional exchange and abbreviate it {\eiff}{ex}.

A végső csereszabály rögzíti az implikáció és az ekvivalencia közötti kapcsolatot. Ezt a szabályt kétfeltételes cserének nevezzük és {\eiff}{ex} -nek rövidítjük.

\begin{center}
\begin{tabular}{rl}
$[(\script{A}\eif\script{B})\eand(\script{B}\eif\script{A})] \Longleftrightarrow (\script{A}\eiff\script{B})$
& {\eiff}{ex}
\end{tabular}
\end{center}


%Although they don't do it in the book, I've been in the habit of writing $(\script{A}\eand\script{B}\eand\script{C})$ and dropping the inner pair of parentheses. This is fine. If we'd wanted to, we could have defined the basic rules in a more general way:

%Noha nem ezt teszik a könyvben, szokásom volt $(\script{A}\eand\script{B}\eand\script{C})$ írása és a belső zárójelek eldobása. . Jól van ez így. Ha szeretnénk, általánosabban meghatározhattuk volna az alapszabályokat:

%\begin{proof}
%	\have[n]{a1}{\script{A}_1}
%	\have{2}{\script{A}_2}
%	\have[\vdots]{1}{\vdots}
%	\have[n]{an}{\script{A}_n}
%	\have[\ ]{aaa}{\script{A}_1~\eand\ldots\eand~\script{A}_n} \ai{}
%\end{proof}

%\bigskip
%\begin{proof}
%	\have{3}{\script{A}_1~\eand\ldots\eand~\script{A}_n}
%	\have{1}{\script{A}_i} \ae{}
%\end{proof}

%\bigskip
%\begin{proof}
%	\have{1}{\script{A}}
%	\have{3}{\script{A}\eor\script{B}_1\eor\script{B}_2\ldots\eor\script{B}_n} \ai{}
%\end{proof}

%We don't need these extended versions, since for any given n we could prove them as a derived rule.

%Nincs szükségünk ezekre a kibővített verziókra, mivel bármely adott n-re származékos szabályként bizonyíthatjuk őket.


\section*{Rules for quantifiers}

\section{A kvantorokra vonatkozó szabályok}

For proofs in QL, we use all of the basic rules of SL plus four new basic rules: both introduction and elimination rules for each of the quantifiers.

A PL bizonyításaihoz a KL összes alapszabályát, valamint négy új alapszabályt használunk: a bevezetési és a megszüntetési szabályokat az egyes kvantorokra.

Since all of the derived rules of SL are derived from the basic rules, they will also hold in QL. We will add another derived rule, a replacement rule called quantifier negation.

Mivel az KL összes származtatott szabálya az alapszabályokból származik, PL-ben is megmaradnak. Felveszünk egy újabb származtatott szabályt, egy helyettesítő szabályt, melyet kvantor tagadásnak hívnak.

\subsection{Substitution instances}

\subsection{Helyettesítési példák}

In order to concisely state the rules for the quantifiers, we need a way to mark the relation between quantified sentences and their instances. For example, the sentence $Pa$ is a particular instance of the general claim $\forall x Px$.

A kvantorokra vonatkozó szabályok rövid ismertetése érdekében, szükségünk van egy módra a kvantifikált mondatok és azok példányai közötti kapcsolat megjelölésére. Például a $Pa$ a $\forall x Px$ általános kijelentés egy bizonyos példánya.

%LLM fordítása vége

%BA fordítása kezdet

For a wff \script{A}, a constant \script{c}, and a variable \script{x}, define a \define{substitution instance} of $\forall \script{x}\script{A}$ or $\exists \script{x}\script{A}$ is the wff that we get by replacing every occurrence of \script{x} in \script{A} with \script{c}. We call \script{c} the \define{instantiating constant}.

Az \script{A} jfk, \script{c} konstans és \script{x} változó esetén $\forall \script{x}\script{A}$ vagy $\exists \script{x}\script{A}$  \define{helyettesítő példányát} úgy definiáljuk, mint a jfk, amelyet akkor kapunk, ha \script{x} minden \script{A} béli előfordulását \script{c}-vel helyettesítjük. \script{c}-t \define{példányosító állandónak} nevezzük.

To underscore the fact that the variable \script{x} is replaced by the instantiating constant \script{c}, we will write the original quantified expressions as $\forall \script{x}\script{A}\script{x}$ and $\exists \script{x}\script{A}\script{x}$. And we will write the substitution instance \script{A}\script{c}.

Annak kihangsúlyozására, hogy az \script{x} változó helyébe a példányosító \script{c} állandó lép, az eredeti, kvantifikált kifejezéseket $\forall \script{x} \script{A} \script{x}$-ként és $\exists \script{x} \script{A} \script {x}$-ként írjuk, és a helyettesítő példányt \script{A}\script{c}-vel írjuk.

Note that \script{A}, \script{x}, and \script{c} are all meta-variables. That is, they are stand-ins for any wff, variable, and constant whatsoever. And when we write \script{A}\script{c}, the constant \script{c} may occur multiple times in the wff \script{A}.

Vegye figyelembe hogy \script{A}, \script{x} és \script{c} mind meta-változók. Vagyis beleillenek bármilyen jfk-ba, változóba és konstansba. Továbbá amikor \script{A}\script{c}-t írunk, a \script{c} állandó többször is előfordulhat az \script{A} jfk-ban.


For example:

Példa:

\begin{itemize}
\item $Aa \eif Ba$, $Af \eif Bf$, and $Ak \eif Bk$ are all substitution instances of $\forall x(Ax \eif Bx)$; the instantiating constants are $a$, $f$, and $k$, respectively.
\item $Raj$, $Rdj$, and $Rjj$ are substitution instances of $\exists zRzj$; the instantiating constants are $a$, $d$, and $j$, respectively.
\end{itemize}

\begin{itemize}
\item $Aa \eif Ba$, $Af \eif Bf$ és $Ak \eif Bk$, helyettesítő példányai $\forall x(Ax \eif Bx)$-nek; a példányosító állandók, $a$, $f$ és $k$.
\item $Raj$, $Rdj$ és $Rjj$, helyettesítő példányai $\exists zRzj$-nek; a példányosító állandók, $a$, $d$ és $j$.
\end{itemize}

\subsection*{Universal elimination}
\subsection{Univerzális elimináció}

If you have $\forall x Ax$, it is legitimate to infer that anything is an $A$. You can infer $Aa$, $Ab$, $Az$, $Ad_3$. You can infer any substitution instance, $A\script{c}$ for any constant \script{c}.

Ha teljesül hogy $\forall x Ax$, akkor jogosan feltételezhető, hogy bármi lehet $A$. Bevezetheti $Aa$-t, $Ab$-t, $Az$-t, $Ad_3$-at, tetszőleges helyettesítő példányt, $A\script{c}$ bármilyen \script{c} állandóra.

This is the general form of the universal elimination rule ($\forall$E):

Az univerzális kvantor eliminációs szabályának általános formája ($\forall$E):

\begin{proof}
	\have[m]{a}{\forall \script{x}\script{A}\script{x}}
	\have[\ ]{c}{\script{A}\script{c}} \Ae{a}
\end{proof}

When using the $\forall$E rule, you write the substituted sentence with the constant \script{c} replacing all occurrences of the variable \script{x} in \script{A}. For example:

A $\forall$E szabály használatakor a helyettesített mondatot a \script{c} állandóval kell írni, amely az  \script{x} változó összes előfordulását felváltja \script{A}-ban. Például:

\begin{proof}
	\hypo{a}{\forall x(Mx \eif Rxd)}
	\have{c}{Ma \eif Rad} \Ae{a}
	\have{d}{Md \eif Rdd} \Ae{a}
\end{proof}


\subsection*{Existential introduction}

\subsection{Egzisztenciális bevezetés}

It is legitimate to infer $\exists x Px$ if you know that \emph{something} is a $P$. It might be any particular thing at all. For example, if you have $Pa$ available in the proof, then $\exists x Px$ follows. 

Jogosan következtethetünk arra, hogy $\exists x Px$, ha tudjuk hogy \emph{valami} $P$. Ez lehet bármilyen konkrét dolog. Például, ha rendelkezésre áll $Pa$ a bizonyításban, akkor $ \exists x Px$ következik.
cica

%BA fordítása vége

%WN fordítása kezdet

This is the existential introduction rule ($\exists$I):

Ez az egzisztenciális bevezetési szabály ($\exists$I):

\begin{proof}
	\have[m]{a}{\script{A}\script{c}}
	\have[\ ]{c}{\exists \script{x}\script{A}\script{x}} \Ei{a}
\end{proof}

It is important to notice that the variable \script{x} does not need to replace all occurrences of the constant \script{c}. You can decide which occurrences to replace and which to leave in place.
For example:

Fontos észrevenni, hogy az \script{x} változóra nem szükségszerű kicserélni a \script{c} konstans összes előfordulását. Eldöntheted, hogy mely előfordulásokat cseréled ki és melyeket hagyod a helyén.
Például: 
\nopagebreak
\begin{proof}
	\hypo{a}{Ma \eif Rad}
	\have{b}{\exists x(Ma \eif Rax)} \Ei{a}
	\have{c}{\exists x(Mx \eif Rxd)} \Ei{a}
	\have{d}{\exists x(Mx \eif Rad)} \Ei{a}
	\have{e}{\exists y\exists x(Mx \eif Ryd)} \Ei{d}
	\have{f}{\exists z\exists y\exists x(Mx \eif Ryz)} \Ei{e}
\end{proof}

\subsection{Universal introduction}
\subsection*{Univerzális bevezetés}
A universal claim like $\forall x Px$ would be proven if {every} substitution instance of it had been proven. That is, if every sentence $Pa$, $Pb$, $\ldots$ were available in a proof, then you would certainly be entitled to claim $\forall x Px$. Alas, there is no hope of proving \emph{every} substitution instance. That would require proving $Pa$, $Pb$, $\ldots$, $Pj_2$, $\ldots$, $Ps_7$, $\ldots$, and so on to infinity. There are infinitely many constants in QL, and so this process would never come to an end.

Egy olyan általános állítás, mint $\forall x Px$ bebizonyosodna, ha {minden} helyettesítési értéke bebizonyosodna. Vagyis, ha minden mondat $Pa$, $Pb$, $\ldots$ rendelkezésre állna bizonyítékként, akkor mindenképp jogosult lenne állítani, hogy $\forall x Px$. Sajnos nincs remény bebizonyítani \emph{minden} helyettesítési értékre. Ehhez szükséges lenne bebizonyítani $Pa$, $Pb$, $\ldots$, $Pj_2$, $\ldots$, $Ps_7$, $\ldots$, és így tovább végtelenig. Végtelen sok konstans van a PL-ben, és ez a folyamat soha nem érne véget.

Consider instead a simple argument: $\forall x Mx$, \therefore\ $\forall y My$

Ehelyett szemléljünk meg egy egyszerű érvet: $\forall x Mx$, \therefore\ $\forall y My$

It makes no difference to the meaning of the sentence whether we use the variable $x$ or the variable $y$, so this argument is obviously valid. Suppose we begin in this way:

Nincs különbség a tétel értelmezésében, akár az $x$ akár az $y$ változót használjuk, tehát ez az érv egyértelműen érvényes.
Tegyük fel, hogy így kezdjük:

\begin{proof}
	\hypo{x}{\forall x Mx} \by{want $\forall y My$}{}
	\have{a}{Ma} \Ae{x}
\end{proof}

We have derived $Ma$. Nothing stops us from using the same justification to derive $Mb$, $\ldots$, $Mj_2$, $\ldots$, $Ms_7$, $\ldots$, and so on until we run out of space or patience. We have effectively shown the way to prove $M\script{c}$ for any constant \script{c}. From this, $\forall y My$ follows.

Származtattunk $Ma$-t. Semmi sem állíthat meg minket, hogy használjuk ugyan ezt az indoklást, hogy ebből származtassunk  $Mb$, $\ldots$, $Mj_2$, $\ldots$, $Ms_7$, $\ldots$, és így tovább, amíg ki nem fogyunk a helyből vagy a türelemből. Valójában megmutattuk a bizonyítási módját $M\script{c}$-nek minden \script{c} konstansra. Ebből következik, hogy $\forall y My$.

%WN fordítása vége

%BB fordítása kezdet

\begin{proof}
	\hypo{x}{\forall x Mx}
	\have{a}{Ma} \Ae{x}
	\have{y}{\forall y My} \Ai{a}
\end{proof}

It is important here that $a$ was just some arbitrary constant. We had not made any special assumptions about it. If $Ma$ were a premise of the argument, then this would not show anything about \emph{all} $y$. For example:

Itt fontos megjegyezni, hogy $a$ egy tetszőleges konstans. Nem szabtunk meg előfeltételt vele kapcsolatban. Ha $Ma$ előfeltétele lenne az argumentumnak, akkor ez semmit sem mondana \emph{minden} $y$-ról. Például:

\begin{proof}
	\hypo{x}{\forall x Rxa}
	\have{a}{Raa} \Ae{x}
	\have{y}{\forall y Ryy} \by{not allowed!}{}
\end{proof}

This is the schematic form of the universal introduction rule ($\forall$I):

Ez az univerzális kvantor bevezetési szabályának vázlatos formája ($\forall$I):

\begin{proof}
	\have[m]{a}{\script{A}\script{c}^\ast}
	\have[\ ]{c}{\forall \script{x}\script{A}\script{x}} \Ai{a}
\end{proof}
$^\ast$ The constant \script{c} must not occur in any undischarged assumption.

$^\ast$ A \script{c} konstans nem fordulhat elő egyik nem teljesített feltevésben sem.

Note that we can do this for any constant that does not occur in an undischarged assumption and for any variable.

Vegyük észre, hogy ezt bármely konstansal vagy változóval meg tudjuk tenni, amely nem fordul elő egyik nem teljesített feltevésben sem.

Note also that the constant may not occur in any \emph{undischarged} assumption, but it may occur as the assumption of a subproof that we have already closed. For example, we can prove $\forall z(Dz \eif Dz)$ without any premises.

Azt is vegyük észre, hogy a konstans bár nem fordulhat elő egyik \emph{nem teljesített} felvetésben sem, de előfordulhat egy felvetés albizonyítékaként, amit már bezártunk. Pédául tudjuk bizonyítani, hogy  $\forall z(Dz \eif Dz)$ bármilyen előfeltétel nélkül.

\begin{proof}
	\open
		\hypo{f1}{Df}\by{want $Df$}{}
		\have{f2}{Df}\by{R}{f1}
	\close
	\have{ff}{Df \eif Df}\ci{f1-f2}
	\have{zz}{\forall z(Dz \eif Dz)}\Ai{ff}
\end{proof}


\subsection*{Existential elimination}
\subsection{Egzisztenciális elimináció}
A sentence with an existential quantifier tells us that there is \emph{some} member of the UD that satisfies a formula. For example, $\exists x Sx$ tells us (roughly) that there is at least one $S$. It does not tell us \emph{which} member of the UD satisfies $S$, however. We cannot immediately conclude $Sa$, $Sf_{23}$, or any other substitution instance of the sentence. What can we do?

Egy egzisztenciális kvantoros mondatból az derül ki, hogy \emph{létezik} egy olyan UD-tag, amely kielégíti a képletet. Például $\exists x Sx$-ból az derül ki (nagyjából), hogy  vagy legalább egy $S$. Azonban nem derül ki, hogy az UD \emph{mely} tagja teljesíti $S$-t. Nem tudunk azonnal következtetést levonni $Sa$, $Sf_{23}$, vagy bármely egyéb helyettesítő példájából. Mit tehetünk?

Suppose that we knew both $\exists x Sx$ and $\forall x(Sx \eif Tx)$. We could reason in this way:

Tételezzük fel, hogy mind az $\exists x Sx$, mind az $\forall x(Sx \eif Tx)$ ismertek. Az alábbi módon érvelhetünk:

%BB fordítása vége

%GM fordítása kezdet

\begin{quote}
Since $\exists x Sx$, there is something that is an $S$. We do not know which constants refer to this thing, if any do, so call this thing `Ishmael'. From $\forall x(Sx \eif Tx)$, it follows that if Ishmael is an $S$, then it is a $T$. Therefore, Ishmael is a $T$.  Because Ishmael is a $T$, we know that $\exists x Tx$.
\end{quote}

\begin{quote}
Miután $\exists x Sx$, van valami dolog, ami $S$. Nem tudjuk, hogy melyik konstansok vonatkoznak erre a dologra, de ha egyáltalán van ilyen, akkor hívjuk ezt „Ishmaelnek”. $\forall x(Sx \eif Tx)$-ből az következik, hogy ha Ishmael egy $S$, akkor egy $T$. Azaz Ishmael egy $T$.  Mivel Ishmael egy $T$, tudjuk, hogy $\exists x Tx$.
\end{quote}

In this paragraph, we introduced a name for the thing that is an $S$. We gave it an arbitrary name (`Ishmael') so that we could reason about it and derive some consequences from there being an $S$. Since `Ishmael' is just a bogus name introduced for the purpose of the proof and not a genuine constant, we could not mention it in the conclusion. Yet we could derive a sentence that does not mention Ishmael; namely, $\exists x Tx$. This sentence does follow from the two premises.

Ebben a bekezdésben, bevezettünk egy nevet a dolognak, ami egy $S$. Adtunk neki egy tetszőleges nevet, ami „Ishmael” volt, hogy tudjunk gondolkodni róla és tudjunk következtetéseket levonni abból, hogy milyen $S$-nek lenni. Mivel „Ishmael” csak egy kitalált név amit a bizonyítás céljából vezettünk be és nem egy valódi konstans, így nem említhettük volna meg a konkluzióban. Mégis származtatni tudtunk egy olyan mondatot, amely nem említi Ishmaelt név szerint:, $\exists x Tx$. Ez a mondat következik a két premisszából.

We want the existential elimination rule to work in a similar way. Yet since English language words like `Ishmael' are not symbols of QL, we cannot use them in formal proofs. Instead, we will use constants of QL which do not otherwise appear in the proof.

Azt akarjuk, hogy az egzisztenciális elimináció szabály hasonlóan működjön. Mégis a magyar nyelvben az olyan szavak, mint az „Ishmael” nem szimbólumai a PL-nek, így nem használhatjuk formális bizonyításokban. Ehelyett a PL konstansait fogjuk használni, amelyek amúgy nem jelennek meg a bizonyitásban.

A constant that is used to stand in for whatever it is that satisfies an existential claim is called a \define{proxy}. Reasoning with the proxy must all occur inside a subproof, and the proxy cannot be a constant that is doing work elsewhere in the proof.

A konstanst ami bármi helyében állhatott, ami kielégít egy egzisztenciális igényt, azt \define{strómannak}-nak hívjuk. A strómannal való érvelésnek egy albizonyításban kell megjelennie, és a stróman nem lehet olyan konstans, ami máshol is szerepel a bizonyításban.

This is the schematic form of the existential elimination rule ($\exists$E): 

Ez a sematikus formája az egzisztenciális eliminációs szabálynak ($\exists$E):

\begin{proof}
	\have[m]{a}{\exists \script{x}\script{A}\script{x}}
	\open	
		\hypo[n]{b}{\script{A}\script{c}^\ast}
		\have[p]{c}{\script{B}}
	\close
	\have[\ ]{d}{\script{B}} \Ee{a,b-c}
\end{proof}
$^\ast$ The constant \script{c} must not appear in $\exists\script{x}\script{A}\script{x}$, in \script{B}, or in any undischarged assumption.

$^\ast$ A \script{c} konstans nem jelenhet meg $\exists\script{x}\script{A}\script{x}$-ben,\script{B}-ben, vagy bármelyik más kielégítetlen felvetésben.


Since the proxy constant is just a place holder that we use inside the subproof, it cannot be something that we know anything particular about. So it cannot appear in the original sentence $\exists\script{x}\script{A}\script{x}$ or in an undischarged assumption. Moreover, we do not learn anything about the proxy constant by using the $\exists$E rule. So it cannot appear in \script{B}, the sentence you prove using $\exists$E.

Mivel a stróman konstans csak egy helyettesítő, amit a részbizonyitásban használunk, ezért nem lehet olyan dolog, amiről bármi részletet tudnunk. Szóval nem jelenhet meg az eredeti kijelentésben. Ráadásul, nem tudunk meg semmit a stróman konstansról azzal, hogy használjuk a $\exists$E szabályt. Szóval nem jelenhet meg a \script{B} mondatban, amit a  $\exists$E használatával bizonyítasz.

The easiest way to satisfy these requirements is to pick an entirely new constant when you start the subproof, and then not to use that constant anywhere else in the proof. Once you close the subproof, do not mention it again.

A legegyszerűbb módja ezen követelmények kielégítésének az, hogy választasz egy teljesen új konstanst, amikor elkezded az albizonyítást és ezután ezt a konstanst nem használod máshol a bizonyításban. Amikor befejezted az albizonyitást, ne említsd meg mégegyszer.

With this rule, we can give a formal proof that $\exists x Sx$ and $\forall x(Sx \eif Tx)$ together entail $\exists x Tx$.

Ezzel a szabállyal adni tudunk egy formális bizonyítást arra, hogy $\exists x Sx$-ből és $\forall x(Sx \eif Tx)$-ből együttesen következik, hogy $\exists x Tx$.

%GM fordítása vége

%GyR fordítása kezdet

\begin{proof}
	\hypo{es}{\exists x Sx}
	\hypo{ast}{\forall x(Sx \eif Tx)}\by{want $\exists x Tx$}{}
	\open
		\hypo{s}{Si}
		\have{st}{Si \eif Ti}\Ae{ast}
		\have{t}{Ti} \ce{s,st}
		\have{et1}{\exists x Tx}\Ei{t}
	\close
	\have{et2}{\exists x Tx}\Ee{es,s-et1}
\end{proof}

\begin{proof}
	\hypo{es}{\exists x Sx}
	\hypo{ast}{\forall x(Sx \eif Tx)}\by{$\exists x Tx$-et szeretnénk}{}
	\open
		\hypo{s}{Si}
		\have{st}{Si \eif Ti}\Ae{ast}
		\have{t}{Ti} \ce{s,st}
		\have{et1}{\exists x Tx}\Ei{t}
	\close
	\have{et2}{\exists x Tx}\Ee{es,s-et1}
\end{proof}

Notice that this has effectively the same structure as the English-language argument with which we began, except that the subproof uses the proxy constant `$i$' rather than the bogus name `Ishmael'.

Vegyük észre, hogy a fenti példának ténylegesen ugyanaz a felépítése, mint a „magyar nyelvő” példának, amivel kezdtünk, attól eltekintve, hogy a részbizonyítás a `$i$' stróman állandót használja a kétes „Ishmael” név helyett.

\subsection{Quantifier negation}
\subsection{Kvantor negáció}


When translating from English to QL, we noted that $\enot\exists x\enot\script{A}$ is logically equivalent to $\forall x\script{A}$. In QL, they are provably equivalent. We can prove one half of the equivalence with a rather gruesome proof:

A magyarról PL-re történő fordítás során megfigyeltük, hogy $\enot\exists x\enot\script{A}$ logikailag ekvivalens ezzel: $\forall x\script{A}$. PL-ben azonban csak valószínűsíthető ez az egyenlőség. Az egyenlőség egyik felét egy meglehetősen bonyolult levezetéssel tudjuk bebizonyítani:

\begin{proof}
	\hypo{Aa}{\forall x Ax} \by{want $\enot\exists x\enot Ax$}{}
	\open
		\hypo{Ena}{\exists x\enot Ax}\by{for reductio}{}
		\open
			\hypo{nc}{\enot Ac}\by{for $\exists$E}{}
			\open
				\hypo{Aa2}{\forall x Ax}\by{for reductio}{}
				\have{c2}{Ac}\Ae{Aa}
				\have{nc2}{\enot Ac}\by{R}{nc}
			\close
			\have{nAa}{\enot\forall x Ax}\ni{Aa2-nc2}
		\close
		\have{Aa3}{\forall x Ax}\by{R}{Aa}
		\have{nAa3}{\enot\forall x Ax}\Ee{Ena,nc-nAa}
	\close
	\have{nEna}{\enot\exists x\enot Ax}\ni{Ena-nAa3}
\end{proof}

\begin{proof}
	\hypo{Aa}{\forall x Ax} \by{$\enot\exists x\enot Ax$-et szeretnénk}{}
	\open
		\hypo{Ena}{\exists x\enot Ax}\by{ellentmondás céljából}{}
		\open
			\hypo{nc}{\enot Ac}\by{$\exists$E-ért}{}
			\open
				\hypo{Aa2}{\forall x Ax}\by{ellentmondás céljából}{}
				\have{c2}{Ac}\Ae{Aa}
				\have{nc2}{\enot Ac}\by{R}{nc}
			\close
			\have{nAa}{\enot\forall x Ax}\ni{Aa2-nc2}
		\close
		\have{Aa3}{\forall x Ax}\by{R}{Aa}
		\have{nAa3}{\enot\forall x Ax}\Ee{Ena,nc-nAa}
	\close
	\have{nEna}{\enot\exists x\enot Ax}\ni{Ena-nAa3}
\end{proof}

In order to show that the two sentences are genuinely equivalent, we need a second proof that assumes $\enot\exists x\enot\script{A}$ and derives $\forall x\script{A}$. We leave that proof as an exercise for the reader.

Ahhoz, hogy rávilágítsunk arra, hogy a két mondat ténylegesen egyenlő egymással, egy második bizonyításra is szükségünk lesz, amely egyrészt feltételezi, hogy $\enot\exists x\enot\script{A}$, és ebből levezeti a $\forall x\script{A}$ összefüggést. Ezt a bizonyítást gyakorlásként az Olvasóra bízzuk.

It will often be useful to translate between quantifiers by adding or subtracting negations in this way, so we add two derived rules for this purpose. These rules are called quantifier negation (QN):

A kvantorok közötti fordítás során gyakran hasznosnak bizonyul, ha ily módon negációkat adunk vagy veszünk el, éppen ezért két további levezetett szabállyal kell kibővíteni a folyamatot. Ezeket a szabályokat kvantor negációnak (QN) nevezzük: 

\begin{center}
\begin{tabular}{rl}
$\enot\forall\script{x}\script{A} \Longleftrightarrow \exists\script{x}\enot\script{A}$\\
$\enot\exists\script{x}\script{A} \Longleftrightarrow \forall\script{x}\enot\script{A}$
& QN
\end{tabular}
\end{center}
Since QN is a replacement rule, it can be used on whole sentences or on subformulae.

Mivel a QN egy helyettesítési szabály, ezért akár egész mondatokon, vagy részképleteken is alkalmazható.

%GyR fordítása vége

%KT fordítása kezdet

\section*{Rules for identity}
\section{Az identitás szabályai}
The identity predicate is not part of QL, but we add it when we need to symbolize certain sentences. For proofs involving identity, we add two rules of proof.

Az identitás predikátum nem része a PL-nek, de amikor bizonyos mondatokat kell szimbolizálnunk, hozzáadjuk. Az identitással kapcsolatos bizonyításokhoz két szabályt adunk hozzá.

Suppose you know that many things that are true of $a$ are also true of $b$. For example: $Aa\eand Ab$, $Ba\eand Bb$, $\enot Ca\eand\enot Cb$, $Da\eand Db$, $\enot Ea\eand\enot Eb$, and so on. This would not be enough to justify the conclusion $a=b$. (See p.~\pageref{model.nonidentity}.) In general, there are no sentences that do not already contain the identity predicate that could justify the conclusion $a=b$. This means that the identity introduction rule will not justify $a=b$ or any other identity claim containing two different constants.

Tegyük fel, hogy számos dolog, amely igaz  $a$-ra, az igaz $b$-re. Például: $Aa\eand Ab$, $Ba\eand Bb$, $\enot Ca\eand\enot Cb$, $Da\eand Db$, $\enot Ea\eand\enot Eb$, és így tovább. Ez nem lenne elegendő $a=b$ következtetés igazolásához. (Lásd a \pageref{model.nonidentity}. oldalt.) Általában nincs olyan mondat, amely nem tartalmazza az identitás predikátumot, amely igazolhatná $a=b$ következtetést. Ez azt jelenti, hogy az identitás bevezetési szabálya nem igazolja $a=b$-t vagy bármely más identitási igényt, amely tartalmaz két különálló állandót.

However, it is always true that $a=a$. In general, no premises are required in order to conclude that something is identical to itself. So this will be the identity introduction rule, abbreviated {=}I:

Azonban $a=a$ mindig igaz. Általában nincs szükség premisszákra annak belátásához, hogy valami azonos önmagával. Ez lesz tehát az identitás bevezetési szabálya, rövidítve {=}I:

\begin{proof}
	\have[\ \,\,\,]{x}{\script{c}=\script{c}} \by{=I}{}
\end{proof}

Notice that the {=}I rule does not require referring to any prior lines of the proof. For any constant \script{c}, you can write $\script{c}=\script{c}$ on any point with only the {=}I rule as justification.

Vegyük észre, hogy az {=}I szabályhoz nem szükségesek a bizonyítás előző sorai. Bármely \script{c} konstansra, bármely ponton felírható, hogy $\script{c}=\script{c}$, erre az {=}I szabály az igazolás.

If you have shown that $a=b$, then anything that is true of $a$ must also be true of $b$. For any sentence with $a$ in it, you can replace some or all of the occurrences of $a$ with $b$ and produce an equivalent sentence. For example, if you already know $Raa$, then you are justified in concluding $Rab$, $Rba$, $Rbb$.

Ha belátjuk, hogy $a=b$, akkor minden, ami igaz $a$-ra, az $b$-re is igaz kell legyen. Bármely $a$-t tartalmazó mondatra igaz, hogy ha $a$ néhány vagy mindegyik előfordulását $b$-re cseréljük, akkor az eredetivel ekvivalens mondatot kapunk. Például, ha $Raa$ már ismert, akkor $Rab$, $Rba$, $Rbb$ következtetések igazoltak.


The identity elimination rule ({=}E) allows us to do this. It justifies replacing terms with other terms that are identical to it.

Az identitás eliminálási szabálya ({=}E) lehetővé teszi ezt. Biztosítja a kifejezések helyettesítését más kifejezésekkel, amelyek azonosak vele.

For writing the rule, we will introduce a new bit of symoblism. For a sentence \script{A} and constants \script{c} and \script{d}, \script{A}{\script{c}$\circlearrowleft$\script{d}} is a sentence produced by replacing some or all instances of \script{c} in \script{A} with \script{d} or replacing instances of \script{d} with \script{c}. This is not the same as a substitution instance, because one constant need not replace every occurrence of the other (although it may).

A szabály megírásához a szimbolizmus egy új részét vezetjük be. Egy \script{A} mondatra és \script{c}, \script{d} konstansokra, \script{A}{\script{c}$\circlearrowleft$\script{d}} egy olyan mondat, amelyet úgy kaptunk meg, hogy \script{c} néhány vagy minden példányának \script{A}-beli előfordulását \script{d} példányaira cseréltük vagy \script{d} példányait \script{c}-re cseréltük. Ez nem ugyanaz, mint egy helyettesítő példány, mert egy konstansnak nem szükséges helyettesítenie a másik minden egyes előfordulását (bár erre lehetőség van).

We can now concisely write {=}E in this way:

{=}E tömören:

\begin{proof}
	\have[m]{e}{\script{c}=\script{d}}
	\have[n]{a}{\script{A}}
	\have[\ ]{ea1}{\script{A}\script{c}\circlearrowleft\script{d}} \by{=E}{e,a}
\end{proof}
\nopagebreak



%The basic rules for conjunction can be valuable in a proof even if there are no conjunctions in any of the assumptions; the basic rules for disjunction can be used even if there are no disjunctions in any assumptions; and similarly for the other basic rules. The rules for identity are different, in that there must be an identity claim in some assumption in order for the rules to do any work. Other than the trivial identity that we can introduce with the {=}I rule


%do not apply we can now prove that identity is \emph{transitive}: If $a=b$ and $b=c$, then $a=c$. The proof proceeds in this way:
%\begin{proof}
%	\open
%		\hypo{p}{a=b \eand b=c}\by{want $a=c$}{}
%		\have{ab}{a=b}\ae{p}
%		\have{bc}{b=c}\ae{p}
%		\have{ac}{a=c}\by{{=}E}{ab,bc}
%	\close
%	\have{conc}{(a=b \eand b=c)\eif a=c} \ci{p-ac}
%\end{proof}


%As an example, consider this argument:
%\begin{quote}
%There is only one button in my pocket. There is a blue button in my pocket. Therefore, there is no button in my pocket that is not blue.
%\end{quote}
%We begin by defining a symbolization key:
%\begin{ekey}
%\item{UD:} buttons in my pocket
%\item{Bx:} $x$ is blue.
%\end{ekey}
%\begin{proof}
%	\hypo{one}{\forall x\forall y\ x=y}
%	\hypo{eb}{\exists x Bx} \by{want $\enot\exists x \enot Bx$}{}
%	\open
%		\hypo{be1}{Be}
%		\have{ef1}{e=f}\Ae{one}
%		\have{bf1}{Bf}\by{{=}E}{ef1,be1}
%	\close
%	\have{bf}{Bf}\Ee{eb,be1-bf1}
%	\have{ab}{\forall x Bx}\Ai{bf}
%	\have{nnab}{\enot\enot\forall x Bx}\by{DN}{ab}
%	\have{nenb}{\enot\exists x\enot Bx}\by{QN}{nnab}
%\end{proof}

To see the rules in action, consider this proof:

Ahhoz, hogy lássuk a szabályt működés közben, vizsgáljuk meg ezt a bizonyítást:

\begin{proof}
	\hypo{one}{\forall x\forall y\ x=y}
	\hypo{eb}{\exists x Bx}
	\hypo{Abnc}{\forall x(Bx \eif \enot Cx)}
		\by{want $\enot\exists x Cx$}{}
	\open
		\hypo{be1}{Be}
		\have{ef1}{\forall y\ e=y}\Ae{one}
		\have{ef2}{e=f}\Ae{ef1}
		\have{bf1}{Bf}\by{{=}E}{ef2,be1}
		\have{bnc1}{Bf\eif\enot Cf}\Ae{Abnc}
		\have{ncf1}{\enot Cf}\ce{bnc1,bf1}
	\close
	\have{cf}{\enot Cf}\Ee{eb,be1-ncf1}
	\have{Anc}{\forall x \enot Cx}\Ai{cf}
	\have{nEc}{\enot\exists x Cx}\by{QN}{Anc}
\end{proof}

%KT fordítása vége

\section{Proof strategy}

There is no simple recipe for proofs, and there is no substitute for practice. Here, though, are some rules of thumb and strategies to keep in mind.

\paragraph{Work backwards from what you want.}
The ultimate goal is to derive the conclusion. Look at the conclusion and ask what the introduction rule is for its main logical operator. This gives you an idea of what should happen \emph{just before} the last line of the proof. Then you can treat this line as if it were your goal. Ask what you could do to derive this new goal.

For example: If your conclusion is a conditional $\script{A}\eif\script{B}$, plan to use the {\eif}I rule. This requires starting a subproof in which you assume \script{A}. In the subproof, you want to derive \script{B}.

%ZsGy fordítása kezdet

\paragraph{Work forwards from what you have.}
When you are starting a proof, look at the premises; later, look at the sentences that you have derived so far. Think about the elimination rules for the main operators of these sentences. These will tell you what your options are.

\paragraph{Dolgozzon előre abból amilye van.}
Amikor egy bizonyítást kezd, nézze meg a feltételeket; majd nézze meg az eddig kapott tételeket. Gondoljon ezeknek a tételeknek a fő műveleteire vonatkozó eliminációs szabályokra. Ezek majd elárulják mik a lehetőségei.

For example: If you have $\forall x\script{A}$, think about instantiating it for any constant that might be helpful. If you have $\exists x\script{A}$ and intend to use the $\exists$E rule, then you should assume $\script{A}[c|x]$ for some $c$ that is not in use and then derive a conclusion that does not contain $c$.

Például: Ha Önnek adott $\forall x\script{A}$, gondoljon a szemléltetésére bármely olyan állandóval amely hasznos lehet. Ha Önnek adott $\exists x\script{A}$ és szándékában áll használni a $\exists$E szabályt, akkor feltételezheti $\script{A}[c|x]$ -t néhány olyan $c$-re amely nincs használatban, ekkor levonhat egy olyan következtetést, amely nem tartalmazza c-t.

For a short proof, you might be able to eliminate the premises and introduce the conclusion. A long proof is formally just a number of short proofs linked together, so you can fill the gap by alternately working back from the conclusion and forward from the premises.

Egy rövid bizonyításnál, talán sikeresen eliminálhatja a feltételeket és bemutathatja a következtetést. Formailag egy hosszú bizonyítás csak sok rövid bizonyítás összekapcsolva, tehát kitöltheti a rést azáltal, hogy felváltva dolgozik. Visszafelé a következtetéstől és előrefelé a feltételektől. 

\paragraph{Change what you are looking at.}
Replacement rules can often make your life easier. If a proof seems impossible, try out some different substitutions.

\paragraph{Minden csak nézőpont kérdése.}
A behelyettesítési szabályok gyakran megkönnyíthetik az életét. Ha egy bizonyítás lehetetlennek tűnik, próbáljon ki más helyettesítéseket.

For example: It is often difficult to prove a disjunction using the basic rules. If you want to show $\script{A}\eor\script{B}$, it is often easier to show $\enot\script{A}\eif\script{B}$ and use the MC rule.

Például: Általában nehéz bebizonyítani egy diszjunkciót az alap szabályok alkalmazásával. Ha meg akarja mutatni, hogy $\script{A}\eor\script{B}$, akkor gyakran inkább egyszerűbb bebizonyítani, hogy $\enot\script{A}\eif\script{B}$ az MC szabály felhasználásával.

Showing $\enot \exists x\script{A}$ can also be hard, and it is often easier to show  $\forall x\enot \script{A}$ and use the QN rule.

Belátni, hogy $\enot \exists x\script{A}$ is nehéz lehet és általában könnyebb megmutatni, hogy $\forall x\enot \script{A}$ és alkalmazni a QN szabályt.

Some replacement rules should become second nature. If you see a negated disjunction, for instance, you should immediately think of DeMorgan's rule.

Néhány behelyettesítési szabály magától értetődővé kell hogy váljon. Ha például egy negált diszjunkcióval találkozik, egyből DeMorgan szabályára kell gondolnia.

\paragraph{Do not forget indirect proof.}
If you cannot find a way to show something directly, try assuming its negation.

\paragraph{Ne feledkezzen meg az indirekt bizonyításról.}
Ha nem tud bebizonyítani valamit direkt módon, próbálja meg feltenni az ellenkezőjét.

Remember that most proofs can be done either indirectly or directly. One way might be easier--- or perhaps one sparks your imagination more than the other--- but either one is formally legitimate.

A legtöbb bizonyítás elvégezhető indirekt és direkt módon is. Az egyik talán könnyebb -- vagy esetleg az egyik jobban mozgatja a képzelőerejét, mint a másik -- de mindkét módszer formálisan elfogadott.

\paragraph{Repeat as necessary.} Once you have decided how you might be able to get to the conclusion, ask what you might be able to do with the premises. Then consider the target sentences again and ask how you might reach them.

\paragraph{Szükség szerint ismételjen.} Miután eldöntötte, hogyan juthat el a következtetésre, kérdezze meg, mit tehetne a feltételekkel. Ezután vegye fontolóra célként kitűzött tételeket, és nézze meg, hogyan érheti el őket.

\paragraph{Persist.}
Try different things. If one approach fails, then try something else.

\paragraph{Legyen kitartó.}
Próbáljon ki különböző dolgokat. Ha egy megközelítés nem sikerül, akkor próbáljon ki valami mást.

%ZsGy fordítása vége

%BBB fordítása kezdet

\section*{Proof-theoretic concepts}
\section{Bizonyításelméleti koncepciók}

We will use the symbol `$\vdash$' to indicate that a proof is possible. This symbol is called the \emph{turnstile}. Sometimes it is called a \emph{single turnstile}, to underscore the fact that this is not the {double turnstile} symbol ($\models$) that we used to represent semantic entailment in ch.~\ref{ch.semantics}.

A „$\vdash$” szimbólummal fogjuk szemléltetni, ha egy bizonyítás lehetséges. Ezt a szimbólumot \emph{sorompónak} hívjuk. Néha \emph{egyszeres sorompónak} hívjuk, ha hangsúlyozni szeretnénk, hogy ez nem a {dupla sorompó} szimbólum ($\models$), amit a szemantikus következetesség (öröklés?) jelölésére használtunk \aref{ch.semantics}. fejezetben.

When we write $\{\script{A}_1,\script{A}_2,\ldots\}\vdash\script{B}$, this means that it is possible to give a proof of \script{B} with $\script{A}_1$,$\script{A}_2$,$\ldots$ as premises. With just one premise, we leave out the curly braces, so $\script{A}\vdash\script{B}$ means that there is a proof of \script{B} with \script{A} as a premise. Naturally, $\vdash\script{C}$ means that there is a proof of \script{C} that has no premises.

Amikor azt írjuk, hogy $\{\script{A}_1,\script{A}_2,\ldots\}\vdash\script{B}$, ez azt jelenti, hogy lehetséges bebizonyítani  \script{B}-t az $\script{A}_1$,$\script{A}_2$,$\ldots$ premisszákkal. Csak egyetlen premisszával kihagyhatjuk a kapcsos zárójeleket, tehát $\script{A}\vdash\script{B}$ azt jelenti, hogy létezik bizonyítás \script{B}-re \script{A} premisszával. Természetesen, a $\vdash\script{C}$ jelentése, hogy \script{C}-nek van egy bizonyítása, aminek nincs premisszája.

Often, logical proofs are called \emph{derivations}. So $\script{A}\vdash\script{B}$ can be read as `\script{B} is derivable from \script{A}.'

Gyakran a logikai bizonyításokat \emph{levezetésnek} nevezzük. Így $\script{A}\vdash\script{B}$ úgy is olvasható, hogy „\script{B} levezethető \script{A}-ból.”

A \define{theorem} is a sentence that is derivable without any premises; i.e., \script{T} is a theorem if and only if $\vdash\script{T}$.

A \define{tétel} egy mondat, ami levezethető premisszák nélkül, például \script{T} egy tétel akkor és csak akkor, ha $\vdash\script{T}$.

It is not too hard to show that something is a theorem--- you just have to give a proof of it. How could you show that something is \emph{not} a theorem? If its negation is a theorem, then you could provide a proof. For example, it is easy to prove $\enot(Pa \eand \enot Pa)$, which shows that $(Pa \eand \enot Pa)$ cannot be a theorem. For a sentence that is neither a theorem nor the negation of a theorem, however, there is no easy way to show this. You would have to demonstrate not just that certain proof strategies fail, but that no proof is possible. Even if you fail in trying to prove a sentence in a thousand different ways, perhaps the proof is just too long and complex for you to make out.

Nem túl nehéz megmutatni, hogy valami egy tétel -- csak be kell bizonyítani. Hogyan tudjuk megmutatni, hogy valami \emph{nem} egy tétel? Ha a negáltja egy tétel, akkor tudjuk bizonyítani. Például könnyű bizonyítani $\enot(Pa \eand \enot Pa)$, ami azt mutatja, hogy $(Pa \eand \enot Pa)$ nem lehet tétel. Egy mondatnál azonban, ami nem tétel, valamint nem egy tétel negáltja, ezt nehéz bebizonyítani. Demonstrálnunk kéne nem csak azt, hogy némely bizonyítási módszerek megbuknak, de azt is, hogy nem létezik semmilyen bizonyítás. Még akkor is, ha ezerféle képpen próbáltuk már bebizonyítani a mondatot, lehetséges, hogy a bizonyítás csak túl hosszú és komplex ahhoz, hogy megértsük.

Two sentences \script{A} and \script{B} are \define{provably equivalent} if and only if each can be derived from the other; i.e., $\script{A}\vdash\script{B}$ and $\script{B}\vdash\script{A}$

Két mondat \script{A} és \script{B} \define{bizonyíthatóan ekvivalens} akkor és csak akkor, ha mindkettő levezethető a másikból, vagyis $\script{A}\vdash\script{B}$ és $\script{B}\vdash\script{A}$

It is relatively easy to show that two sentences are provably equivalent--- it just requires a pair of proofs. Showing that sentences are \emph{not} provably equivalent would be much harder. It would be just as hard as showing that a sentence is not a theorem. (In fact, these problems are interchangeable. Can you think of a sentence that would be a theorem if and only if \script{A} and \script{B} were provably equivalent?)

Viszonylag könnyű megmutatni, hogy két mondat bizonyíthatóan ekvivalens -- csak egy pár bizonyításra van szükség. Megmutatni azt, hogy mondatok \emph{nem} bizonyíthatóan ekvivalensek, sokkal nehezebb. Ugyanannyira nehéz, mint megmutatni, hogy egy mondat nem tétel. (Valójában ezek a problémák felcserélhetőek. Tudna egy olyan mondatra gondolni, ami egy tétel akkor és csak akkor, ha \script{A} és \script{B} bizonyíthatóan ekvivalens?)

The set of sentences $\{\script{A}_1,\script{A}_2,\ldots\}$ is \define{provably inconsistent} if and only if a contradiction is derivable from it; i.e., for some sentence \script{B}, $\{\script{A}_1,\script{A}_2,\ldots\}\vdash\script{B}$ and $\{\script{A}_1,\script{A}_2,\ldots\}\vdash\enot \script{B}$.

Egy mondathalmaz $\{\script{A}_1,\script{A}_2,\ldots\}$ \define{bizonyíthatóan inkonzisztens} akkor és csak akkor, ha egy ellentmondás vezethető le belőle, például valamely mondatra \script{B}, $\{\script{A}_1,\script{A}_2,\ldots\}\vdash\script{B}$ és $\{\script{A}_1,\script{A}_2,\ldots\}\vdash\enot \script{B}$.

It is easy to show that a set is provably inconsistent: You just need to assume the sentences in the set and prove a contradiction. Showing that a set is \emph{not} provably inconsistent will be much harder. It would require more than just providing a proof or two; it would require showing that proofs of a certain kind are \emph{impossible}.

Könnyű megmutatni, hogy egy halmaz bizonyíthatóan inkonzisztens: Csak vegyük a mondatokat a halmazból és bizonyítsunk be egy ellentmondást. Megmutatni azt, hogy egy halmaz \emph{nem} bizonyíthatóan inkonzisztens, sokkal nehezebb. Több szükséges hozzá mint egy-kettő bizonyítás, szükséges lenne megmutatni, hogy némely bizonyítások \emph{lehetetlenek}.

%BBB fordítása vége

%GT fordítása kezdet




\section{Proofs and models}
As you might already suspect, there is a connection between \emph{theorems} and \emph{tautologies}.

There is a formal way of showing that a sentence is a theorem: Prove it. For each line, we can check to see if that line follows by the cited rule. It may be hard to produce a twenty line proof, but it is not so hard to check each line of the proof and confirm that it is legitimate--- and if each line of the proof individually is legitimate, then the whole proof is legitimate. Showing that a sentence is a tautology, though, requires reasoning in English about all possible models. There is no formal way of checking to see if the reasoning is sound. Given a choice between showing that a sentence is a theorem and showing that it is a tautology, it would be easier to show that it is a theorem.

Contrawise, there is no formal way of showing that a sentence is \emph{not} a theorem. We would need to reason in English about all possible proofs. Yet there is a formal method for showing that a sentence is not a tautology. We need only construct a model in which the sentence is false. Given a choice between showing that a sentence is not a theorem and showing that it is not a tautology, it would be easier to show that it is not a tautology.

Fortunately, a sentence is a theorem if and only if it is a tautology. If we provide a proof of $\vdash\script{A}$ and thus show that it is a theorem, it follows that \script{A} is a tautology; i.e., $\models\script{A}$. Similarly, if we construct a model in which \script{A} is false and thus show that it is not a tautology, it follows that \script{A} is not a theorem.

In general, $\script{A}\vdash\script{B}$ if and only if $\script{A}\models\script{B}$. As such:
\begin{itemize}
\item An argument is \emph{valid} if and only if \emph{the conclusion is derivable from the premises}.
\item Two sentences are \emph{logically equivalent} if and only if they are \emph{provably equivalent}.
\item A set of sentences is \emph{consistent} if and only if it is \emph{not provably inconsistent}.
\end{itemize}
You can pick and choose when to think in terms of proofs and when to think in terms of models, doing whichever is easier for a given task. Table \ref{table.ProofOrModel} summarizes when it is best to give proofs and when it is best to give models.

In this way, proofs and models give us a versatile toolkit for working with arguments. If we can translate an argument into QL, then we can measure its logical weight in a purely formal way. If it is deductively valid, we can give a formal proof; if it is invalid, we can provide a formal counterexample.

%GT fordítása vége

%HHB  fordítása kezdet

\begin{table}[h!]
\begin{center}
\begin{tabular*}{\textwidth}{p{10em}|p{10em}|p{10em}|}
\cline{2-3}


 & {\centerline{YES}} & {\centerline{NO}}\\
\cline{2-3}
 & {\centerline{IGEN}} & {\centerline{NEM}}\\
\cline{2-3}


Is \script{A} a tautology? & prove $\vdash\script{A}$ & give a model in which \script{A} is false\\
\cline{2-3}
Az \script{A} tautológia? & bizonyítsa be, hogy $\vdash\script{A}$ & adjon meg egy példát, amiben \script{A} hamis\\
\cline{2-3}


Is \script{A} a contradiction? & prove $\vdash\enot\script{A}$ & give a model in which \script{A} is true\\
\cline{2-3}
Az \script{A} ellentmondás? & bizonyítsa be, hogy $\vdash\enot\script{A}$ & adjon meg egy példát, amiben \script{A} igaz\\
\cline{2-3}


Is \script{A} contingent? & give a model in which \script{A} is true and another in which \script{A} is false & prove $\vdash\script{A}$ or $\vdash\enot\script{A}$\\
\cline{2-3}
Az \script{A} kontingens? & adjon meg egy példát, amiben \script{A} igaz és egy másikat, amiben \script{A} hamis & bizonyítsa be, hogy $\vdash\script{A}$ vagy $\vdash\enot\script{A}$\\
\cline{2-3}


Are \script{A} and \script{B} equivalent? & prove \mbox{$\script{A}\vdash\script{B}$} and \mbox{$\script{B}\vdash\script{A}$}  & give a model in which \script{A} and \script{B} have different truth values\\
\cline{2-3}
Az \script{A} és a \script{B} ekvivalensek? & bizonyítsa be, hogy \mbox{$\script{A}\vdash\script{B}$} és \mbox{$\script{B}\vdash\script{A}$}  & adj meg egy példát, amiben \script{A} és \script{B} különböző igazságértékkel rendelkeznek\\
\cline{2-3}


Is the set \model{A} consistent? & give a model in which all the sentences in \model{A} are true & taking the sentences in \model{A}, prove \script{B} and \enot\script{B}\\
\cline{2-3}
Az \model{A} halmaz konzisztens? & adjon meg egy modellt, ahol \model{A} minden eleme igaz & \model{A} elemeiből kiindulva  bizonyítsa be, hogy \script{B} és \enot\script{B}\\
\cline{2-3}


Is the argument \mbox{`\script{P}, \therefore\ \script{C}'} valid? & prove $\script{P}\vdash\script{C}$ & give a model in which \script{P} is true and \script{C} is false\\
\cline{2-3}
A \mbox{`\script{P}, \therefore\ \script{C}'} érvelés valós? & bizonyítás $\script{P}\vdash\script{C}$ & adjon meg egy modellt, amiben \script{P} igaz, és \script{C} hamis\\
\cline{2-3}


\end{tabular*}
\end{center}

\caption{Sometimes it is easier to show something by providing proofs than it is by providing models. Sometimes it is the other way round.  It depends on what you are trying to show.}
\caption{Néha egyszerűbb úgy megmutatni valamit, hogy bizonyítékokat mutatunk be és nem modelleket. Néha az ellenkezője igaz. Attól függ, hogy mit szeretnénk bemutatni.}

\label{table.ProofOrModel}
\end{table}


\section{Soundness and completeness}
\section{Megbízhatóság és teljesség}

This toolkit is incredibly convenient. It is also intuitive, because it seems natural that provability and semantic entailment should agree. Yet, do not be fooled by the similarity of the symbols `$\models$' and `$\vdash$.' The fact that these two are really interchangeable is not a simple thing to prove.

Ez az eszközkészlet hihetetlenül megfelelő. Továbbá intuitív, mert természetesnek tűnik, hogy a bizonyíthatóságnak és a szemantikai következtetéseknek meg kell egyezniük. Mégis, ne hagyjuk magunkat átverni a szimbólumok hasonlóságával `$\models$' and `$\vdash$. A tény, hogy ez a két szimbólum valóban felcserélhető, nem egyszerű bizonyítani.

Why should we think that an argument that \emph{can be proven} is necessarily a \emph{valid} argument? That is, why think that $\script{A}\vdash\script{B}$ implies $\script{A}\models\script{B}$?

Miért kéne azt gondolnunk, hogy egy érv, amit \emph{bizonyítani lehet} feltétlenül \emph{helytálló} érv-e? Tehát, miért gondoljuk, hogy $\script{A}\vdash\script{B}$ ugyan azt jelenti, mint $\script{A}\models\script{B}$?

This is the problem of \define{soundness}. A proof system is \define{sound} if there are no proofs of invalid arguments. Demonstrating that the proof system is sound would require showing that \emph{any} possible proof is the proof of a valid argument. It would not be enough simply to succeed when trying to prove many valid arguments and to fail when trying to prove invalid ones.

Ez a probléma az \define{megbízhatósággal}. Egy bizonyítási rendszer \define{megbízható}, ha nincsenek bizonyítékok érvénytelen érvekre. Annak bemutatása érdekében, hogy a bizonyítási rendszer megbízható szükséges megmutatni, hogy \emph{minden} lehetséges bizonyíték egy valós érv bizonyítéka. Nem lenne elég, ha csak az lenne sikeres, amikor próbáljuk bebizonyítani, hogy sok valós érvelés van és nem sikerülne bizonyítani a valótlanokat.

Fortunately, there is a way of approaching this in a step-wise fashion. If using the {\eand}E rule on the last line of a proof could never change a valid argument into an invalid one, then using the rule many times could not make an argument invalid. Similarly, if using the {\eand}E and {\eor}E rules individually on the last line of a proof could never change a valid argument into an invalid one, then using them in combination could not either.

Szerencsére, van egy módja annak, hogy lépésről-lépésre megközelítsük. Ha az {\eand}E szabályt használjuk az utolsó soron, akkor a bizonyíték soha nem tudja megváltoztatni a valós érvet valótlanra, utána, ha a szabályt többször alkalmazzuk egy érvet nem tud érvénytelenné tenni. Hasonlóan, ha az {\eand}E és a {\eor}E szabályokat egyszerre használjuk az utolsó vonalán egy érvnek soha nem tudja megváltoztatni az érvényességét érvénytelenre, ahogyan, ha kombinálva használjuk őket, az sem tud változtatni.

%HHB fordítása vége

%NG fordítása kezdet
The strategy is to show for every rule of inference that it alone could not make a valid argument into an invalid one. It follows that the rules used in combination would not make a valid argument invalid. Since a proof is just a series of lines, each justified by a rule of inference, this would show that every provable argument is valid.

A módszer lényege, hogy megmutassuk, minden levezetési szabály esetén a szabály egymaga nem elég ahhoz, hogy egy helyes érvelést meghamisítson. Ebből adódik, hogy a szabályok kombinálása sem vezet egy helyes érvelés meghamisitásához. Mivel a bizonyitás csak sorok sorozata amelyeket mind egy levezetési szabály támaszt alá, ez elegendő annak a bebizonytáshoz, hogy minden bebizonyítható érvelés helyes.

Consider, for example, the {\eand}I rule. Suppose we use it to add \script{A}\eand\script{B} to a valid argument. In order for the rule to apply, \script{A} and \script{B} must already be available in the proof. Since the argument so far is valid, \script{A} and \script{B} are either premises of the argument or valid consequences of the premises. As such, any model in which the premises are true must be a model in which \script{A} and \script{B} are true. According to the definition of \define{truth in QL}, this means that \script{A}\eand\script{B} is also true in such a model. Therefore, \script{A}\eand\script{B} validly follows from the premises. This means that using the {\eand}E rule to extend a valid proof produces another valid proof.

Példa képpen vegyünk az {\eand}I szabályt. Tegyük fel, hogy ezek segítségével hozzáadjuk \script{A}\eand\script{B}-t egy helyes érveléshez. Ez csak akkor működik, ha már \script{A} és \script{B} része az érvelésnek. Mivel az érvelés eddig helyes volt, ezért \script{A} és \script{B} vagy az érvelés premisszái, vagy a premisszák helyes koknluziói. Vagyis minden modellben ahol a premisszák igazak, abban a modellben \script{A} és \script{B} is igazak. Tehát \script{A}\eand\script{B}-ből következik, hogy az {\eand}E szabályt alkalmazva egy igaz bizonyításból egy újabb igaz bizonyítás következik.

In order to show that the proof system is sound, we would need to show this for the other inference rules. Since the derived rules are consequences of the basic rules, it would suffice to provide similar arguments for the 16 other basic rules. This tedious exercise falls beyond the scope of this book.

Ahhoz, hogy megbizonyosodjunk, hogy a bizonyítási rendszerünk konzisztens ez előbbi folyamatot alkalmazunk kell a többi levezetési szabályra is. Mivel a deriválási szabályok konzekvensek az alap szabályokkal elegendő bebizonyítanunk, hogy a többi 16 alap szabály is ugyan így viselkednek. De ezek bizonyítása meghaladja e könyvnek a célkitűzését.

Given a proof that the proof system is sound, it follows that every theorem is a tautology.

Ha be van bizonyítva, hogy a bizonyítási rendszer konzisztens, következik, hogy minden tétel egyben tautológia is.

It is still possible to ask: Why think that \emph{every} valid argument is an argument that can be proven?  That is, why think that $\script{A}\models\script{B}$ implies $\script{A}\vdash\script{B}$?

Még mindig meg lehet kérdezni: Miért is gondoljuk, hogy \emph{minden} igaz állítás az egy bebizonyítható állítás? Vagyis miért gondoljuk, hogy  $\script{A}\models\script{B}$-ből következik $\script{A}\vdash\script{B}$?

This is the problem of \define{completeness}. A proof system is \define{complete} if there is a proof of every valid argument. Completeness for a language like QL was first proven by Kurt G\"odel in 1929. The proof is beyond the scope of this book.

Ez a \define{teljesség} problémája. Egy bizonyítási rendszer akkor \define{teljes}, ha valamennyi igaz állításra találunk bizonyítást. A QL nyelv teljességét először Kurt Gödel bizonyította be 1929-ben. A bizonyítással ámbár ez a könyv nem kíván foglalkozni.

The important point is that, happily, the proof system for QL is both sound and complete. This is not the case for all proof systems and all formal languages. Because it is true of QL, we can choose to give proofs or construct models--- whichever is easier for the task at hand.

A fontos pont, hogy szerencsére a bizonyítási rendzer a QL nyelvre egyben konzisztens és teljes. Ez nem minden bizonyítási rendszerről valamint nyelvről mondható el. De mivel a QL rendszerre igaz, választhatunk, hogy bizonyításokat adunk, vagy modelleket készítünk -- amelyik éppen könnyebb az adott feladatban.

\section*{Summary of definitions}
\begin{itemize}
\item A sentence \script{A} is a \define{theorem} if and only if $\vdash\script{A}$.

\item Two sentences \script{A} and \script{B} are \define{provably equivalent} if and only if $\script{A}\vdash\script{B}$ and $\script{B}\vdash\script{A}$.

\item $\{\script{A}_1,\script{A}_2,\ldots\}$ is \define{provably inconsistent} if and only if, for some sentence \script{B}, $\{\script{A}_1,\script{A}_2,\ldots\}\vdash(\script{B} \eand \enot \script{B})$.
\end{itemize}

\section*{Definiciók összegzése}
\begin{itemize}
\item Az \script{A} mondat akkor és csakis akkor egy \define{tétel}, ha $\vdash\script{A}$.

\item Két mondtat \script{A} és \script{B} akkor \define{bizonyíthatóan ekvivalensek}, ha $\script{A}\vdash\script{B}$ és $\script{B}\vdash\script{A}$.

\item $\{\script{A}_1,\script{A}_2,\ldots\}$ \define{bizonyíthatóan inkonzisztens} akkor és csakis akkor, ha egy \script{B} mondatra, $\{\script{A}_1,\script{A}_2,\ldots\}\vdash(\script{B} \eand \enot \script{B})$.
\end{itemize}

%NG fordítása vége

%TS fordítása kezdet

\practiceproblems

\solutions
\problempart
\label{pr.justifySLproof}
Provide a justification (rule and line numbers) for each line of proof that requires one.
* A rész Indokoljon (szabály és sorszám) minden egyes állítást, amit szükséges.
\begin{multicols}{2}
\begin{proof}
\hypo{1}{W \eif \enot B}
\hypo{2}{A \eand W}
\hypo{2b}{B \eor (J \eand K)}
\have{3}{W}{}
\have{4}{\enot B} {}
\have{5}{J \eand K} {}
\have{6}{K}{}
\end{proof}

\begin{proof}
\hypo{1}{L \eiff \enot O}
\hypo{2}{L \eor \enot O}
\open
	\hypo{a1}{\enot L}
	\have{a2}{\enot O}{}
	\have{a3}{L}{}
	\have{a4}{\enot L}{}
\close
\have{3}{L}{}
\end{proof}

\begin{proof}
\hypo{1}{Z \eif (C \eand \enot N)}
\hypo{2}{\enot Z \eif (N \eand \enot C)}
\open
	\hypo{a1}{\enot(N \eor  C)}
	\have{a2}{\enot N \eand \enot C} {}
	\open
		\hypo{b1}{Z}
		\have{b2}{C \eand \enot N}{}
		\have{b3}{C}{}
		\have{b4}{\enot C}{}
	\close
	\have{a3}{\enot Z}{}
	\have{a4}{N \eand \enot C}{}
	\have{a5}{N}{}
	\have{a6}{\enot N}{}
\close
\have{3}{N \eor C}{}
\end{proof}
\end{multicols}

\solutions
\problempart
\label{pr.solvedSLproofs}
Give a proof for each argument in SL.
* B rész Igazoljon minden egyes állítást KL-ben.
\begin{earg}
\item $K\eand L$, \therefore $K\eiff L$
\item $A\eif (B\eif C)$, \therefore $(A\eand B)\eif C$
\item $P \eand (Q\eor R)$, $P\eif \enot R$, \therefore $Q\eor E$
\item $(C\eand D)\eor E$, \therefore $E\eor D$
\item $\enot F\eif G$, $F\eif H$, \therefore $G\eor H$
\item $(X\eand Y)\eor(X\eand Z)$, $\enot(X\eand D)$, $D\eor M$ \therefore $M$
\end{earg}

\problempart
Give a proof for each argument in SL.
C rész Igazoljon minden egyes állítást KL-ben.
\begin{earg}
\item $Q\eif(Q\eand\enot Q)$, \therefore\ $\enot Q$
\item $J\eif\enot J$, \therefore\ $\enot J$
\item $E\eor F$, $F\eor G$, $\enot F$, \therefore\ $E \eand G$
\item $A\eiff B$, $B\eiff C$, \therefore\ $A\eiff C$
\item $M\eor(N\eif M)$, \therefore\ $\enot M \eif \enot N$
\item $S\eiff T$, \therefore\ $S\eiff (T\eor S)$
\item $(M \eor N) \eand (O \eor P)$, $N \eif P$, $\enot P$, \therefore\ $M\eand O$
\item $(Z\eand K) \eor (K\eand M)$, $K \eif D$, \therefore\ $D$
\end{earg}



\problempart
Show that each of the following sentences is a theorem in SL.
D rész Mutassa meg, hogy a következő mondatok mindegyike egy tétel KL-ben.
\begin{earg}
\item $O \eif O$
\item $N \eor \enot N$
\item $\enot(P\eand \enot P)$
\item $\enot(A \eif \enot C) \eif (A \eif C)$
\item $J \eiff [J\eor (L\eand\enot L)]$
\end{earg}

\problempart
Show that each of the following pairs of sentences are provably equivalent in SL.
E rész Mutassa meg, hogy a következő mondatpárok bizonyíthatóan ekvivalensek KL-ben.
\begin{earg}
\item $\enot\enot\enot\enot G$, $G$
\item $T\eif S$, $\enot S \eif \enot T$
\item $R \eiff E$, $E \eiff R$
\item $\enot G \eiff H$, $\enot(G \eiff H)$
\item $U \eif I$, $\enot(U \eand \enot I)$
\end{earg}

\problempart
Provide proofs to show each of the following.
F rész Bizonyítsa a következőket.
\begin{earg}
\item $M \eand (\enot N \eif \enot M) \vdash (N \eand M) \eor \enot M$
\item \{$C\eif(E\eand G)$, $\enot C \eif G$\} $\vdash$ $G$
\item \{$(Z\eand K)\eiff(Y\eand M)$, $D\eand(D\eif M)$\} $\vdash$ $Y\eif Z$
\item \{$(W \eor X) \eor (Y \eor Z)$, $X\eif Y$, $\enot Z$\} $\vdash$ $W\eor Y$
\end{earg}



\problempart
For the following, provide proofs using only the basic rules. The proofs will be longer than proofs of the same claims would be using the derived rules.
G rész A következőknél, az alapszabályok segítségével adjon bizonyítást. A bizonyítás így hosszabb lesz, mint ugyanaz a származtatott szabályok alkalmazásával.

\begin{earg}
\item Show that MT is a legitimate derived rule. Using only the basic rules, prove the following: \script{A}\eif\script{B}, \enot\script{B}, \therefore\ \enot\script{A}
\item Mutassa meg, hogy MT egy legitim származtatott szabály. Csak az alapszabályok használatával, bizonyítsa a következőt: \script{A}\eif\script{B}, \enot\script{B}, \therefore\ \enot\script{A}
\item Show that Comm is a legitimate rule for the biconditional. Using only the basic rules, prove that $\script{A}\eiff\script{B}$ and $\script{B}\eiff\script{A}$ are equivalent.
\item Mutassa meg, hogy Comm egy legitim szabály az ekvivalenciára. Csak az alapszabályok használatával, bizozonyítsa, hogy $\script{A}\eiff\script{B}$ és $\script{B}\eiff\script{A}$ evivalensek.
\item Using only the basic rules, prove the following instance of DeMorgan's Laws: $(\enot A \eand \enot B)$, \therefore\ $\enot(A \eor B)$
\item Csak az alapszabályok használatával, bizonyítsa a következő példán keresztül a DeMorgan törvényeket: $(\enot A \eand \enot B)$, \therefore\ $\enot(A \eor B)$
\item Without using the QN rule, prove $\enot\exists x\enot\script{A} \vdash \forall x\script{A}$
\item QN szabály használata nélkül, bizonyítsa $\enot\exists x\enot\script{A} \vdash \forall x\script{A}$
\item Show that {\eiff}{ex} is a legitimate derived rule. Using only the basic rules, prove that $D\eiff E$ and $(D\eif E)\eand(E\eif D)$ are equivalent.
\item Mutassa meg, hogy {\eiff}{ex} egy legitim származtatott szabály. Csak az alapszabályok használatával bizonyítsa, hogy $D\eiff E$ és $(D\eif E)\eand(E\eif D)$ ekvivalensek.
\end{earg}


\solutions
\problempart
\label{pr.subinstanceQL}
\begin{earg}
\item Identify which of the following are substitution instances of $\forall x Rcx$: $Rac$, $Rca$, $Raa$, $Rcb$, $Rbc$, $Rcc$, $Rcd$, $Rcx$
\item Határozza meg, hogy a következők közül melyik helyettesítheti $\forall x Rcx$: $Rac$, $Rca$, $Raa$, $Rcb$, $Rbc$, $Rcc$, $Rcd$, $Rcx$
\item Identify which of the following are substitution instances of $\exists x\forall y Lxy$:
$\forall y Lby$, $\forall x Lbx$, $Lab$, $\exists x Lxa$
\item Határozza meg, hogy a kövezkezők közül melyik helyettesítheti $\exists x\forall y Lxy$:
$\forall y Lby$, $\forall x Lbx$, $Lab$, $\exists x Lxa$
\end{earg}



\newpage
\solutions
\problempart
\label{pr.justifyQLproof}
Provide a justification (rule and line numbers) for each line of proof that requires one.
* I rész Indokoljon (szabály és sorszám) minden egyes állítást, amit szükséges.
\begin{multicols}{2}
%$\{\forall x(\exists y)(Rxy \eor Ryx),\forall x\enot Rmx\}\vdash\exists xRxm$
\begin{proof}
\hypo{p1}{\forall x\exists y(Rxy \eor Ryx)}
\hypo{p2}{\forall x\enot Rmx}
\have{3}{\exists y(Rmy \eor Rym)}{}
	\open
		\hypo{a1}{Rma \eor Ram}
		\have{a2}{\enot Rma}{}
		\have{a3}{Ram}{}
		\have{a4}{\exists x Rxm}{}
	\close
\have{n}{\exists x Rxm} {}
\end{proof}

%$\{\forall x(\exists yLxy \eif \forall zLzx), Lab\} \vdash \forall xLxx$
\begin{proof}
\hypo{1}{\forall x(\exists yLxy \eif \forall zLzx)}
\hypo{2}{Lab}
\have{3}{\exists y Lay \eif \forall zLza}{}
\have{4}{\exists y Lay} {}
\have{5}{\forall z Lza} {}
\have{6}{Lca}{}
\have{7}{\exists y Lcy \eif \forall zLzc}{}
\have{8}{\exists y Lcy}{}
\have{9}{\forall z Lzc}{}
\have{10}{Lcc}{}
\have{11}{\forall x Lxx}{}
\end{proof}


% $\{\forall x(Jx \eif Kx), \exists x\forall y Lxy, \forall x Jx\} \vdash \exists x(Kx \eand Lxx)$
\begin{proof}
\hypo{a}{\forall x(Jx \eif Kx)}
\hypo{b}{\exists x\forall y Lxy}
\hypo{c}{\forall x Jx}
\open
	\hypo{2}{\forall y Lay}
	\have{d}{Ja}{}
	\have{e}{Ja \eif Ka}{}
	\have{f}{Ka}{}
	\have{3}{Laa}{}
	\have{4}{Ka \eand Laa}{}
	\have{5}{\exists x(Kx \eand Lxx)}{}
\close
\have{j}{\exists x(Kx \eand Lxx)}{}
\end{proof}





%$\vdash \exists x Mx \eor \forall x\enot Mx$
\begin{proof}
	\open
		\hypo{p1}{\enot (\exists x Mx \eor \forall x\enot Mx)}
		\have{p2}{\enot \exists x Mx \eand \enot \forall x\enot Mx}{}
		\have{p3}{\enot \exists x Mx}{}
		\have{p4}{\forall x\enot Mx}{}
		\have{p5}{\enot \forall x\enot Mx}{}
	\close
\have{n}{\exists x Mx \eor \forall x\enot Mx} {}
\end{proof}
\end{multicols}

\solutions
\problempart
\label{pr.someQLproofs}
Provide a proof of each claim.
* J rész Bizonyítson be minden egyes állítást.
\begin{earg}
\item $\vdash \forall x Fx \eor \enot \forall x Fx$
\item $\{\forall x(Mx \eiff Nx), Ma\eand\exists x Rxa\}\vdash \exists x Nx$
\item $\{\forall x(\enot Mx \eor Ljx), \forall x(Bx\eif Ljx), \forall x(Mx\eor Bx)\}\vdash \forall xLjx$
\item $\forall x(Cx \eand Dt)\vdash \forall xCx \eand Dt$
\item $\exists x(Cx \eor Dt)\vdash \exists x Cx \eor Dt$
\end{earg}

\problempart
Provide a proof of the argument about Billy on p.~\pageref{surgeon2}.
K rész Bizonyitsa Billy érvelését az o.~\pageref{surgeon2}.


%TS fordítása vége

%BBB fordítása kezdet

\problempart
\label{pr.BarbaraEtc.proof1}
Look back at Part \ref{pr.BarbaraEtc} on p.~\pageref{pr.BarbaraEtc}. Provide proofs to show that each of the argument forms is valid in QL.

Lásd \ref{pr.BarbaraEtc} rész a \pageref{pr.BarbaraEtc}. oldalon. Bizonyítsa be, hogy az argumentumok mindegyik változata érvényes a PL-ben.


\problempart
\label{pr.BarbaraEtc.proof2}
Aristotle and his successors identified other syllogistic forms. Symbolize each of the following argument forms in QL and add the additional assumptions `There is an $A$' and `There is a $B$.' Then prove that the supplemented arguments forms are valid in QL.

M rész Arisztotelész és utódjai egyéb szillogisztikus formulákat is azonosítottak. Jelöljük a kötvetkező argumentumok mindegyikét a PL-ben és egészítsük ki azzal a feltevéssel, hogy „van egy $A$” és „van egy $B$”. Ez után bizonyítsuk be, hogy a kiegészített érvelések helyesek a PL-ben.


\begin{description}
\item[Darapti:] All $A$s are $B$s. All $A$s are $C$s.
	\therefore\  Some $B$ is $C$.
\item[Felapton:] No $B$s are $C$s. All $A$s are $B$s.
	\therefore\  Some $A$ is not $C$.
\item[Barbari:] All $B$s are $C$s. All $A$s are $B$s.
	\therefore\  Some $A$ is $C$.
\item[Camestros:] All $C$s are $B$s. No $A$s are $B$s.
	\therefore\  Some $A$ is not $C$.
\item[Celaront:] No $B$s are $C$s. All $A$s are $B$s.
	\therefore\  Some $A$ is not $C$.
\item[Cesaro:] No $C$s are $B$s. All $A$s are $B$s.
	\therefore\  Some $A$ is not $C$.
\item[Fapesmo:] All $B$s are $C$s. No $A$s are $B$s.
	\therefore\  Some $C$ is not $A$.
\end{description}

\begin{description}
\item[Darapti:] Minden $A$ az $B$. Minden $A$ az $C$.
	\therefore\  Néhány $B$ az $C$.
\item[Felapton:] Egyik $B$ sem $C$. Minden $A$ az $B$.
	\therefore\  Néhány $A$ az nem $C$.
\item[Barbari:] Minden $B$ az $C$. Minden $A$ az $B$.
	\therefore\  Néhány $A$ az $C$.
\item[Camestros:] Minden $C$ az $B$. Egyik $A$ sem $B$.
	\therefore\  Néhány $A$ az nem $C$.
\item[Celaront:] Egyik $B$ sem $C$. Minden $A$ az $B$.
	\therefore\  Néhány $A$ az nem $C$.
\item[Cesaro:] Egyik $C$ sem $B$. Minden $A$ az $B$.
	\therefore\  Néhány $A$ az nem $C$.
\item[Fapesmo:] Minden $B$ az $C$. Egyik $A$ sem $B$.
	\therefore\  Néhány $C$ az nem $A$.
\end{description}


\problempart
Provide a proof of each claim.

N rész Bizonyítsa mindegyik állítást.

\begin{earg}
\item $\forall x \forall y Gxy\vdash\exists x Gxx$
\item $\forall x \forall y (Gxy \eif Gyx) \vdash \forall x\forall y (Gxy \eiff Gyx)$
\item $\{\forall x(Ax\eif Bx), \exists x Ax\} \vdash \exists x Bx$
\item $\{Na \eif \forall x(Mx \eiff Ma), Ma, \enot Mb\}\vdash \enot Na$
\item $\vdash\forall z (Pz \eor \enot Pz)$
\item $\vdash\forall x Rxx\eif \exists x \exists y Rxy$
\item $\vdash\forall y \exists x (Qy \eif Qx)$
\end{earg}



\problempart
Show that each pair of sentences is provably equivalent.

O rész Mutassa meg, hogy mindegyik mondatpár bizonyíthatóan ekvivalens.

\begin{earg}
\item $\forall x (Ax\eif \enot Bx)$, $\enot\exists x(Ax \eand Bx)$
\item $\forall x (\enot Ax\eif Bd)$, $\forall x Ax \eor Bd$
\item $\exists x Px \eif Qc$, $\forall x (Px \eif Qc)$
\end{earg}



\problempart
Show that each of the following is provably inconsistent.

P rész Mutassa meg, hogy a következők mindegyike bizonyíthatóan inkonzisztens.

\begin{earg}
\item \{$Sa\eif Tm$, $Tm \eif Sa$, $Tm \eand \enot Sa$\}
\item \{$\enot\exists x Rxa$, $\forall x \forall y Ryx$\}
\item \{$\enot\exists x \exists y Lxy$, $Laa$\}
\item \{$\forall x(Px \eif Qx)$, $\forall z(Pz \eif Rz)$, $\forall y Py$, $\enot Qa \eand \enot Rb$\}
\end{earg}



\solutions
\problempart
\label{pr.likes}
Write a symbolization key for the following argument, translate it, and prove it:

* Q rész Írjon fel egy szimbolizációs kulcsot a következő érvelésre, majd fordítsa le és bizonyítsa:

\begin{quote}
There is someone who likes everyone who likes everyone that he likes. Therefore, there is someone who likes himself.
\end{quote}

\begin{quote}
Van valaki, aki szeret mindenkit, aki szeret mindenkit, akit ő szeret. Tehát, van valaki aki önmagát szereti.
\end{quote}




\problempart
\label{pr.identity}
Provide a proof of each claim.

R rész Bizonyítsa mindegyik állítást.

\begin{earg}
\item $\{Pa \eor Qb, Qb \eif b=c, \enot Pa\}\vdash Qc$
\item $\{m=n \eor n=o, An\}\vdash Am \eor Ao$
\item $\{\forall x x=m, Rma\}\vdash \exists x Rxx$
\item $\enot \exists x x \neq m \vdash \forall x\forall y (Px \eif Py)$
\item $\forall x\forall y(Rxy \eif x=y)\vdash Rab \eif Rba$
\item $\{\exists x Jx, \exists x \enot Jx\}\vdash \exists x \exists y\ x\neq y$
\item $\{\forall x(x=n \eiff Mx), \forall x(Ox \eor \enot Mx)\}\vdash On$
\item $\{\exists x Dx, \forall x(x=p \eiff Dx)\}\vdash Dp$
\item $\{\exists x\bigl[Kx \eand \forall y(Ky \eif x=y) \eand Bx\bigr], Kd\}\vdash Bd$
\item $\vdash Pa \eif \forall x(Px \eor x \neq a)$
\end{earg}



\problempart
Look back at Part \ref{pr.QLarguments} on p.~\pageref{pr.QLarguments}. For each argument: If it is valid in QL, give a proof. If it is invalid, construct a model to show that it is invalid.

S rész Lásd \ref{pr.QLarguments} rész a \pageref{pr.QLarguments}. oldalon. Minden érvelésre: Ha érvényes PL-ben, bizonyítsa. Ha nem igaz, akkor szerkesszen olyan modellt, ami megmutatja, hogy helytelen.

\solutions
\problempart
\label{pr.QLequivornot}
For each of the following pairs of sentences: If they are logically equivalent in QL, give proofs to show this. If they are not, construct a model to show this.

* T rész  Minden mondatpárra: Ha logikailag ekvivalensek PL-ben, akkor bizonyítsa is be. Ha nem ekvivalensek, akkor szerkesszen modellt, ami ezt megmutatja.

\begin{earg}
\item $\forall x Px \eif Qc$, $\forall x (Px \eif Qc)$
\item $\forall x Px \eand Qc$, $\forall x (Px \eand Qc)$
\item $Qc \eor \exists x Qx$, $\exists x (Qc \eor Qx)$
\item $\forall x\forall y \forall z Bxyz$, $\forall x Bxxx$
\item $\forall x\forall y Dxy$, $\forall y\forall x Dxy$
\item $\exists x\forall y Dxy$, $\forall y\exists x Dxy$
\end{earg}

\solutions
\problempart
\label{pr.QLvalidornot}
For each of the following arguments: If it is valid in QL, give a proof. If it is invalid, construct a model to show that it is invalid.

* U rész Minden alábbi kifejezésre: Ha PL-ben helyes, bizonyítsa. Ha helytelen, szerkesszen modellt, mely megmutatja, hogy helytelen.

\begin{earg}
\item $\forall x\exists y Rxy$, \therefore\ $\exists y\forall x Rxy$
\item $\exists y\forall x Rxy$, \therefore\ $\forall x\exists y Rxy$
\item $\exists x(Px \eand \enot Qx)$, \therefore\ $\forall x(Px \eif \enot Qx)$
\item $\forall x(Sx \eif Ta)$, $Sd$, \therefore\ $Ta$
\item $\forall x(Ax\eif Bx)$, $\forall x(Bx \eif Cx)$, \therefore\ $\forall x(Ax \eif Cx)$
\item $\exists x(Dx \eor Ex)$, $\forall x(Dx \eif Fx)$, \therefore\ $\exists x(Dx \eand Fx)$
\item $\forall x\forall y(Rxy \eor Ryx)$, \therefore\ $Rjj$
\item $\exists x\exists y(Rxy \eor Ryx)$, \therefore\ $Rjj$
\item $\forall x Px \eif \forall x Qx$, $\exists x \enot Px$, \therefore\ $\exists x \enot Qx$
\item $\exists x Mx \eif \exists x Nx$, $\enot \exists x Nx$, \therefore\ $\forall x \enot Mx$
\end{earg}



\problempart
\begin{earg}
\item If you know that $\script{A}\vdash\script{B}$, what can you say about $(\script{A}\eand\script{C})\vdash\script{B}$? Explain your answer.
\item If you know that $\script{A}\vdash\script{B}$, what can you say about $(\script{A}\eor\script{C})\vdash\script{B}$? Explain your answer.
\end{earg}

\problempart
\begin{earg}
\item Ha tudjuk, hogy  $\script{A}\vdash\script{B}$, akkor mit mondhatunk arról, hogy $(\script{A}\eand\script{C})\vdash\script{B}$? Válaszát magyarázza!
\item Ha tudjuk, hogy  $\script{A}\vdash\script{B}$, akkor mit mondhatunk arról, hogy $(\script{A}\eor\script{C})\vdash\script{B}$? Válaszát magyarázza!
\end{earg}



%BBB fordítása vége

